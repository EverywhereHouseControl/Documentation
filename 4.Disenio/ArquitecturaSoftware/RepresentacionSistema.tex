\chapter{Representaci\'on del sistema}
En este documento, se describen las distintas arquitecturas adoptadas para realizar un sistema que cumpla con los requisitos solicitados con la m\'axima calidad posible. Podemos representar el sistema como un conjunto de los siguientes componentes:
 \begin{itemize}
 \item Vista de casos de uso: donde se presentan los actores y los casos de uso para el sistema donde se manifiesta lo que percibe los distintos actores en distintas situaciones del sistema. 
 \item Vista l\'ogica: se detallan los requerimientos funcionales y podemos observar como puede funcionar el sistema a trav\'es de diagramas entidad-relaci\'on y diagramas de clase.
 \item Vista de procesos: en esta componente, nos centramos en otros aspectos del sistemas m\'as enfocados al rendimiento y disponibilidad del sistema. Algunos de los tem\'as tratados son la concurrencia, distribuci\'on e integridad del sistema.
 \item Vista de despliegue: en esta vista nos centramos m\'as en mostrar la arquitectura hardware del sistema. Debido al uso de distintos componentes hardware y a sus respectivas interfaces, esta vista se convierte en una de las principales vistas que define el sistema.
 \item Vista de implementaci\'on: describe la estructura general del modelo de implementaci\'on, distribuci\'on del software y la interrelaci\'on entre sus partes.
 \end{itemize}
