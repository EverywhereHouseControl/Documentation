\subsection{Control de dispositivos}
\begin{center}
    \begin{tabular}{|p{3cm}|p{4cm}|p{4cm}|p{4cm}|}
    \hline \multicolumn{3}{|p{9cm}|}{\textit{Control de dispositivos.}} & \textit{Versi\'on 1.0} \\
	\hline \textit{Codigo} & \textit{Fecha} & \multicolumn{2}{|p{6cm}|}{\textit{Autor}} \\
	CU-02 & 05/01/2014 & \multicolumn{2}{|p{6cm}|}{Tirado, Colin} \\		
    \hline \textit{Descripci\'on} & \multicolumn{3}{|p{9cm}|}{El usuario podr\'a modificar el estado de un dispositivo de forma remota.} \\
    \hline \textit{Actores} & \multicolumn{3}{|p{9cm}|}{Super usuario y usuario} \\
    \hline \textit{Entrada} & \multicolumn{3}{|p{9cm}|}{Interfaz gr\'afica y uno de estos par\'ametros:
    \begin{enumerate}
    \item Num\'ericos: para regular la intensidad de una luz, ajustar una temperatura...
    \item Booleano: para apagar o encender un dispositivo.
    \end{enumerate}} \\
    \hline \textit{Precondic\'ion} & \multicolumn{3}{|p{9cm}|}{Conexi\'on establecida con el sistema.} \\
    \hline \textit{Acciones} & \multicolumn{3}{|p{9cm}|}{
    \begin{enumerate}
    \item El usuario se identifica en el sistema EHC.
    \item Tras identificarse, acceder\'a a la secci\'on \textit{Gesti\'on} de la aplicaci\'on.
    \item Selecciona el entorno EHC donde se encuentra el dispositivo.
    \item Selecciona la habitaci\'on donde se encuentra el dispositivo.
    \item Selecciona el elemento a controlar.
    \item Realiza la acci\'on sobre el elemento a controlar.
    \item Confirmar acci\'on.
    \end{enumerate}
       } \\
    \hline \textit{Postcondici\'on} & \multicolumn{3}{|p{9cm}|}{Se registra el evento en la base de datos.} \\
    \hline \textit{Salida} & \multicolumn{3}{|p{9cm}|}{Se notificar\'a al usuario el resultado de la operaci\'on y en caso satisfactorio, se realizar\'a la acci\'on solicitada por el usuario.} \\ \hline
    \end{tabular}
\end{center}