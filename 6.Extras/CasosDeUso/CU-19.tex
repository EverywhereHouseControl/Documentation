\begin{center}
    \begin{tabular}{|p{3cm}|p{4cm}|p{4cm}|p{4cm}|}
    \hline \multicolumn{3}{|p{9cm}|}{\textit{Eliminar un entorno EHC.}} & \textit{Versi\'on 1.0} \\
	\hline \textit{Codigo} & \textit{Fecha} & \multicolumn{2}{|p{6cm}|}{\textit{Autor}} \\
	CU-19 & 02/01/2014 & \multicolumn{2}{|p{6cm}|}{Tirado, Colin} \\		
    \hline \textit{Descripci\'on} & \multicolumn{3}{|p{9cm}|}{El administrador podr\'a eliminar un entorno EHC.} \\
    \hline \textit{Actores} & \multicolumn{3}{|p{9cm}|}{Administrador.} \\
    \hline \textit{Entrada} & \multicolumn{3}{|p{9cm}|}{Interfaz gr\'afica.} \\
    \hline \textit{Precondic\'ion} & \multicolumn{3}{|p{9cm}|}{Conexi\'on establecida con el sistema.} \\
    \hline \textit{Acciones} & \multicolumn{3}{|p{9cm}|}{
        \begin{enumerate}
        \item El administrador se identifica en el sistema EHC.
        \item Tras identificarse, acceder\'a a la secci\'on \textit{Configuraci\'on} de la aplicaci\'on.
        \item Selecciona el bot\'on de eliminar situado sobre el entorno de la lista de entornos mostrados.
        \item Seleccionar el bot\'on de eliminar.
        \item Confirmar cambios.
        \end{enumerate}
           } \\
    \hline \textit{Postcondici\'on} & \multicolumn{3}{|p{9cm}|}{Los cambios estar\'an reflejados en la base de datos.} \\
    \hline \textit{Salida} & \multicolumn{3}{|p{9cm}|}{Se notificar\'a al administrador el resultado de la operaci\'on.} \\ \hline
    \end{tabular}
\end{center}