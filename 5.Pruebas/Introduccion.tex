\section{Introducción}
\subsection{Proposito}
	Este plan de pruebas para el sistema EHC trata de cumplir con los siguientes requisitos:
	\begin{itemize}
	\item Identificar información del proyecto y elementos que deben ser testados.
	\item Listar requisito principales a testear.
	\item Definir una serie de estrategias/reglas para llevar a cabo las pruebas.
	\item Listar los prototipos entregables del proceso de pruebas.
	\end{itemize}
\subsection{Entorno}
	El proyecto que se va evaluar con este plan de pruebas se corresponde con la aplicacion EHC y todo el sistema hardware que la acompaña.
\subsection{Alcance}
	Se realizaran los siguientes tipos de pruebas software:
	\begin{itemize}
	\item Pruebas de funcionalidad.
	\item Pruebas de interfaz de usuario.
	\item Pruebas de la Base de Datos.
	\item Pruebas de Carga y Rendimiento.
	\item Pruebas de Seguridad/Acceso.
	\end{itemize}
	En hardware:
	\begin{itemize}
	\item Probar modulos de forma idependiente.
	\item Probar conjunto de modulos.
	\item Pruebas de rendimiento y stress.
	\item Controles de Seguridad.
	\end{itemize}
\subsection{Vision General}
La siguiente tabla muestra los documentos disponibles y utilizados para este plan de pruebas.

\begin{center}
   \begin{tabular}{|p{3cm}|p{2.2cm}|p{1.8cm}|p{2cm}|p{3cm}|}
     \hline
     \textbf{Documentos} & \textbf{Disponible} & \textbf{Revisado} & \textbf{Autor} & \textbf{Notas} \\ \hline
     Especificacion de Requisitos & SI & SI &  &  \\ \hline
     Especificacion Casos de Uso & SI & SI &  &  \\ \hline
     Especificaciones de Diseño &  &  & & \\ \hline
     Plan de \newline Proyecto &  &  & & \\ \hline
     Prototipos & NO & NO & & \\ \hline
     Manuales de \newline Usuario & NO & NO & & \\ \hline
   \end{tabular}
\end{center}
	
	
	
	
	
	
	
	
	
	
	
	
	