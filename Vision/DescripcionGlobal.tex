\chapter{Descripci�n global del producto}

\section{Perspectiva del producto}
	El producto a desarrollar es un sistema que gestiona y automatiza los recursos de un entorno EHC a trav�s de un dispositivo m�vil de forma remota. 

\section{Resumen de caracter�sticas}
	A continuaci�n se mostrar� un listado de beneficios que obtendr� el cliente a partir del producto: \\ \par 
	\begin{tabular}{|p{8cm}|p{8cm}|}
		\hline \textbf{Beneficio del cliente} & \textbf{Caracter�sticas que lo apoyan} \\
		\hline & \\
		\hline
	\end{tabular}

\section{Suposiciones y dependencias}
	A continuaci�n se muestra una lista de suposiciones y dependencias del sistema que afectan al sistema y al cumplimiento de de lo especificado en los documentos del proyecto: \\ \par 
	\begin{tabular}{|p{4cm}|p{6cm}|p{6cm}|}
		\hline \textbf{Categor�a} &  \textbf{Comentario} & \textbf{Limitaci�n si no se cumple} \\
		\hline Suposici�n & El usuario tiene en el hogar conexi�n a Internet & El usuario solo podr� usar el sistema  EHC de forma local. \\
		\hline Dependencia & El hogar dispone de forma ininterrumpida corriente el�ctrica & El sistema EHC no estar� disponible en esos periodos sin corriente el�ctrica \\
		\hline & & \\
		\hline
	\end{tabular}

\section{Costo y precio}
	TBD

\section{Licencia e instalaci�n}
	La instalaci�n y configuraci�n inicial ser� realizada por el personal de la empresa. Dicha instalaci�n y configuraci�n no solo se centra en el montaje de los elementos hardware necesarios para el funcionamiento del sistema, sino tambi�n en la configuraci�n de la red de comunicaci�n entre los distintos dispositivos que se encuentran en el entorno EHC. \par
	
	Tras dicha instalaci�n, se entregar� al usuario una licencia de uso del sistema EHC con sus correspondientes claves de identificaci�n y uso del sistema.
