\chapter{Requisitos de usuario}
    En esta sección es una lista enumerada de los requisitos de usuario.\newline

    ***********************\newline
    TODO Hay que unificar las dos seccions que vienen a continuacion\newline
    ***********************\newline

\section{Funciones del producto [Esto son los Requisitos de usuario OTRA VEZ]}

    El sistema EHC permitirá la realización de las siguientes funciones:

    \begin{enumerate}
        \item Técnico software/Administrador de usuarios: El técnico del sistema podrá gestionar los usuarios (agregar, modificar, eliminar, buscar, etc.), asimismo resolverá cualquier inconveniente o error que se le pueda presentar al usuario a nivel software.

        \item Control de aparatos: A través de este sistema se podrán manipular/manejar multitud de aparatos electrónicos y eléctricos presentes en el hogar (o centro de trabajo).

        \begin{itemize}
            \item Control remoto: Se tendrá el control de estos aparatos tanto dentro como fuera del lugar en el que esté implantado el sistema EHC.
        \end{itemize}

        \item Perfiles de mandatos: Atendiendo a las posibles necesidades y exigencias del usuario, este tendrá la posibilidad de agrupar una serie o conjunto de mandatos/tareas y configurarlos en distintos perfiles de ejecución. (ej.: En época de vacaciones, gracias a estos perfiles de mandatos el usuario podrá viajar tranquilo, el sistema se encargará de realizar las tareas configuradas en un determinado perfil, llamado por ejemplo 'vacaciones', con ello se conseguirá simular que hay alguien en su casa).

        \item Restricciones/Privilegios de perfil: Posibilidad de crear multitud de perfiles. Un determinado perfil podrá tener acotados sus permisos, funcionalidades y restricciones por motivos de seguridad y protección de datos.(ej.: Un invitado, si quiere el usuario, no podrá tener libre acceso a todas la funcionalidades que presta el sistema en su casa o un empleado podrá tener restringido el acceso a determinadas zonas de un área de trabajo)

        \item Estadísticas de uso: El sistema llevará un informe con las estadísticas de uso de las funcionalidades del propio sistema.

        \item Alertas: El usuario estará completamente informado sobre el estado del sistema y el desarrollo de sus tareas, ya sea por la configuración que haga el usuario de las mismas como por iniciativa del propio sistema.
    \end{enumerate}

\section{Requisitos de usuario [Listado completo]}
    \begin{enumerate}
        %Estos serían, toscamente, las “funcionalidades” que decidimos integrar en el sistema cuando hicimos el estudio de viabilidad. Federico pide que concretemos mas.
            \item Control remoto de TV modelo X mediante codigos del mando modelo Y. Se ofrecera una funcionalidad basica (parcial) del mando a distancia.

            \item Captación, géstion y reporte de informacion relativa a sensores meteorológicos (meteorología: lluvia, viento, sol, etc)

            \item Control por voz de parte del sistema /*Habría que especificar que vamos a poder controlar por voz*/

            \item Reporte de información del sistema en formato audible.

            \item Control y reporte de información del telefonillo modelo X

            \item Reporte de información web relativa a la prevision meteorológica de la zona elegida por el usuario.

            \item Captación y reporte de infomración relativa al consumo energético en tiempo real (factura de la luz)

            \item Control de sistema de climatización modelo X.

            \item Programacion de eventos (grabar tele, despertadores, subir persianas, luces, etc)

            \item Reporte de avisos programables por el usuario (Sistema de alarmas)

            \item Control de subida y bajada de la puerta del garaje con sistema de reporte del estado actual de las misma

            \item Control de iluminazion por zonas a elección del usuario.

            \item Control de subida y bajada de persianas motorizadas con sistema de reporte del estado actual de las mismas.

            \item Control de subida y bajada de toldos motorizados con sistema de reporte del estado actual de las mismas.

            \item Control  y reporte de infomacion de sensores de movimiento

            \item Gestión y reporte de historial de dispositivos controlados por el sistema

            \item Control de regleta  de enchufes

            \item Control y reporte de informacion del sistema de riego automatico

            \item Edición gráfica de representación en planta de la casa controlada

            \item Control y reporte de información del sensotr de humo

            \item Control y reporte de información de camaras de vigilancia

            \item Control de "cosas" mediante perfiles de mandatos

            \item Control y reporte de la programación ofrecida en al TV modelo X

            \item Control de la aplicacion en modo offline cuando el usuario se encuentra dentro de la casa domotizada

    \end{enumerate}
