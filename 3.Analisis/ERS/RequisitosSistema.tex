\chapter{Requisitos de sistema}
    En esta secci�n se muestra los distintos requisitos que deben de cumplir el sistema. La estructura de esta secci�n ser� la siguiente:
    \begin{description}
        \item[Requisitos funcionales:]{Define una funci�n del sistema como un conjunto de entradas, acciones y salidas.}
        \item[Requisitos no funcionales:]{Requisito que especifica los criterios que pueden usarse para juzgar la operaci�n del sistema en lugar de sus comportamientos espec�ficos.}
        \item
    \end{description}

    Las secciones de los requisitos suelen someterse a cambios hasta la entrega del producto, por lo que es conveniente registrar las versiones, autores y fechas del mismo.

    Este documento ser� �til como punto de partida para el equipo de desarrollo con la elaboraci�n de cada requisito, ya que se define el entorno de cada requisito.

\section{Requisitos funcionales}
    \begin{tabular}{|p{3cm}|p{4cm}|p{4cm}|p{4cm}|}
\hline \multicolumn{3}{|p{9cm}|}{\textit{Acceso a la aplicaci\'on}} & \textit{Versi\'on 1.0} \\
\hline \textit{Codigo} & \textit{Fecha} & \textit{Autor} & \textit{Grado de necesidad} \\
RF-01 & 09/12/2013 & Tirado, Colin & Alto \\
\hline \textit{Descripci\'on} & \multicolumn{3}{|p{9cm}|}{Identificaci\'on del usuario en el sistema.} \\
\hline \textit{Entrada} & \multicolumn{3}{|p{9cm}|}{Nombre de usuario y contrase\~na del usuario.} \\
\hline \textit{Precondici\'on} & \multicolumn{3}{|p{9cm}|}{Conexi\'on a la red.} \\
\hline \textit{Acciones} & \multicolumn{3}{|p{9cm}|}{Insertar acciones de forma enumerada} \\
\hline \textit{Postcondici\'on} & \multicolumn{3}{|p{9cm}|}{Se contempla dos situaciones:
\begin{enumerate}
\item Si se identifica con \'exito: se volcar\'a la informaci\'on correspondiente del usuario.
\item En caso contrario: no se aplica.
\end{enumerate}
} \\
\hline \textit{Salida} & \multicolumn{3}{|p{9cm}|}{Se contemplan tres situaciones
\begin{enumerate}
\item Si el usuario no existe: se notificar\'a que el usuario intruducido no est\'a registrado en el sistema.
\item Si se ha introducido de forma err\'onea la contrase\~na: se notificar\'a que se ha introducido mal la contrase\~na.
\item Si se introduce el usuario y contrase\~na correctamente: se mostrar\'a la ventana principal de la aplicaci\'on.
\end{enumerate} } \\ \hline
\end{tabular}
\begin{center}
    \begin{tabular}{|p{2.6cm}|p{12cm}|}
    \hline
    \textbf{Identificador} & RF-02\\
    \hline
    \textbf{Descripcion} & Captacion de informacion relativa a la programación ofrecida en la TV modelo X\\
    \hline
    \textbf{Necesidad} & ESENCIAL, DESEABLE, OPCIONAL\\
    \hline
    \textbf{Dificultad} & ALTA, MEDIA, BAJA\\
    \hline
    \textbf{Prioridad} & ALTA, MEDIA, BAJA\\
    \hline
    \end{tabular}
\end{center}

\begin{center}
    \begin{tabular}{|p{2.6cm}|p{12cm}|}
    \hline
    \textbf{Identificador} & RF-03\\
    \hline
    \textbf{Descripcion} & Captación de informacion relativa a sensores meteorológicos (meteorología: lluvia, viento, sol, etc)\\
    \hline
    \textbf{Necesidad} & ESENCIAL, DESEABLE, OPCIONAL\\
    \hline
    \textbf{Dificultad} & ALTA, MEDIA, BAJA\\
    \hline
    \textbf{Prioridad} & ALTA, MEDIA, BAJA\\
    \hline
    \end{tabular}
\end{center}

\begin{tabular}{|p{3cm}|p{4cm}|p{4cm}|p{4cm}|}
\hline \multicolumn{3}{|p{9cm}|}{\textit{Modificaci\'on del perfil}} & \textit{Versi\'on 1.0} \\
\hline \textit{Codigo} & \textit{Fecha} & \textit{Autor} & \textit{Grado de necesidad} \\
RF-04 & 10/12/2013 & Tirado, Colin & Alto \\
\hline \textit{Descripci\'on} & \multicolumn{3}{|p{9cm}|}{El usuario podr\'a modificar su perfil EHC.} \\
\hline \textit{Entrada} & \multicolumn{3}{|p{9cm}|}{El usuario selecciona el bot\'on correspondiente para modificar su perfil situado en el perfil.} \\
\hline \textit{Precondici\'on} & \multicolumn{3}{|p{9cm}|}{El usuario est\'a identificado en el sistema EHC} \\
\hline \textit{Acciones} & \multicolumn{3}{|p{9cm}|}{
\begin{enumerate}
\item El usuario se identifica en el sistema EHC.
\item Tras identificarse, acceder\'a a la secci\'on \textit{Perfil} de la aplicaci\'on
\item Selecciona el bot\'on de modificar
\end{enumerate}
} \\
\hline \textit{Postcondici\'on} & \multicolumn{3}{|p{9cm}|}{Si se ha realizado alg\'un cambio correctamente, la nueva informaci\'on introducida ser\'a almacenada en nuestra base de datos.} \\
\hline \textit{Salida} & \multicolumn{3}{|p{9cm}|}{Se muestra el perfil del usuario y se notifica si se ha realizado correctamente el cambio.} \\ \hline
\end{tabular}

\begin{center}
    \begin{tabular}{|p{2.6cm}|p{12cm}|}
    \hline
    \textbf{Identificador} & RF-05\\
    \hline
    \textbf{Descripcion} & Control del telefonillo modelo X\\
    \hline
    \textbf{Necesidad} & DESEABLE, OPCIONAL\\
    \hline
    \textbf{Dificultad} & ALTA\\
    \hline
    \textbf{Prioridad} & MEDIA, BAJA\\
    \hline
    \end{tabular}
\end{center}

\begin{center}
    \begin{tabular}{|p{2.6cm}|p{12cm}|}
    \hline
    \textbf{Identificador} & RF-06\\
    \hline
    \textbf{Descripcion} & Control del telefono fijo modelo X\\
    \hline
    \textbf{Necesidad} & OPCIONAL\\
    \hline
    \textbf{Dificultad} & ALTA\\
    \hline
    \textbf{Prioridad} & BAJA\\
    \hline
    \end{tabular}
\end{center}

\begin{center}
    \begin{tabular}{|p{2.6cm}|p{12cm}|}
    \hline
    \textbf{Identificador} & RF-07\\
    \hline
    \textbf{Descripcion} & Captación de informacion web relativa a la prevision meteorológica de la zona elegida por el usuario\\
    \hline
    \textbf{Necesidad} & DESEABLE\\
    \hline
    \textbf{Dificultad} & BAJA\\
    \hline
    \textbf{Prioridad} & MEDIA\\
    \hline
    \end{tabular}
\end{center}

\begin{tabular}{|p{3cm}|p{3cm}|p{3cm}|p{3cm}|}
\hline \multicolumn{3}{|p{9cm}|}{\textit{Consulta metereol\'ogica en el entorno EHC.}} & \textit{Versi\'on 1.0} \\
\hline \textit{Codigo} & \textit{Fecha} & \textit{Autor} & \textit{Grado de necesidad} \\
RF-08 & 10/12/2013 & Tirado, Colin & Medio \\
\hline \textit{Descripci\'on} & \multicolumn{3}{|p{9cm}|}{El usuario podr\'a consultar la metereolog\'ia en el entorno EHC para as\'i, poder realizar alguna acci\'on concreta.} \\
\hline \textit{Entrada} & \multicolumn{3}{|p{9cm}|}{El usuario selecciona el bot\'on de \textit{Consulta metereol\'ogica} situado en la ventana de \textit{Gesti\'on}.} \\
\hline \textit{Precondici\'on} & \multicolumn{3}{|p{9cm}|}{El usuario est\'a identificado en el sistema EHC y tiene configurado un entorno EHC.} \\
\hline \textit{Acciones} & \multicolumn{3}{|p{9cm}|}{
\begin{enumerate}
\item El usuario se identifica en el sistema EHC.
\item Tras identificarse, acceder\'a a la secci\'on \textit{Gesti'on} de la aplicaci\'on.
\item Selecciona el bot\'on de \textit{Consulta metereol\'ogica}.
\end{enumerate}
} \\
\hline \textit{Postcondici\'on} & \multicolumn{3}{|p{9cm}|}{No aplica.} \\
\hline \textit{Salida} & \multicolumn{3}{|p{9cm}|}{Se mostrar\'a la informaci\'on metereol\'ogica del entorno EHC.} \\ \hline
\end{tabular}
\begin{center}
    \begin{tabular}{|p{2.6cm}|p{12cm}|}
    \hline
    \textbf{Identificador} & RF-09\\
    \hline
    \textbf{Descripcion} & Control del sistema de climatización modelo X\\
    \hline
    \textbf{Necesidad} & ESENCIAL\\
    \hline
    \textbf{Dificultad} & MEDIA\\
    \hline
    \textbf{Prioridad} & ALTA\\
    \hline
    \end{tabular}
\end{center}

\begin{center}
    \begin{tabular}{|p{2.6cm}|p{12cm}|}
    \hline
    \textbf{Identificador} & RF-06\\
    \hline
    \textbf{Descripcion} & Control de subida y bajada de la puerta del garaje con sistema de reporte del estado actual de las misma\\
    \hline
    \textbf{Necesidad} & DESEABLE\\
    \hline
    \textbf{Dificultad} & ALTA\\
    \hline
    \textbf{Prioridad} & BAJA\\
    \hline
    \end{tabular}
\end{center}

\begin{center}
    \begin{tabular}{|p{3cm}|p{4cm}|p{4cm}|p{4cm}|}
    \hline \multicolumn{3}{|p{9cm}|}{\textit{Control del tel\'efono fijo}} & \textit{Versi\'on 1.0} \\
    \hline \textit{Codigo} & \textit{Fecha} & \textit{Autor} & \textit{Grado de necesidad} \\
    RF-11 & 10/12/2013 & Tirado, Colin & Medio \\
    \hline \textit{Descripci\'on} & \multicolumn{3}{|p{9cm}|}{El usuario podr\'a controlar desde la aplicaci\'on EHC el tel\'efono fijo, pudiendo recibir o realizar llamadas.} \\
    \hline \textit{Entrada} & \multicolumn{3}{|p{9cm}|}{Se contempla dos situaciones
    \begin{enumerate}
    \item Si desea realizar una llamada: el usuario selecciona el bot\'on correspondiente situado en la zona donde se encuentra dicho elemento en el entorno EHC.
    \item Si recibe una llamada: el usuario selecciona el aviso de la aplicaci\'on EHC y selecciona si desea aceptar o rechazar la llamada.
    \end{enumerate}} \\
    \hline \textit{Precondici\'on} & \multicolumn{3}{|p{9cm}|}{
    \begin{itemize}
    \item El usuario est\'a identificado en el sistema EHC
    \item Tiene configurado un entorno EHC.
    \item Tiene instalado un tel\'efono fijo en el entorno EHC
    \end{itemize}
    } \\
    \hline \textit{Acciones} & \multicolumn{3}{|p{9cm}|}{
    Para el caso querer realizar una llamada:
    \begin{enumerate}
    \item El usuario se identifica en el sistema EHC.
    \item Tras identificarse, acceder\'a a la secci\'on \textit{Gesti\'on} de la aplicaci\'on.
    \item Selecciona la zona del entorno EHC donde desea controlar el elemento.
    \item Realiza la acci\'on sobre el elemento a controlar.
    \end{enumerate}

    Si desea aceptar/rechazar una llamada entrante
    \begin{enumerate}
    \item El usuario recibe un aviso de una llamada entrante a su entorno EHC.
    \item El usuario selecciona si desea rechazar o aceptar una llamada entrante.
    \end{enumerate}
    } \\
    \hline \textit{Postcondici\'on} & \multicolumn{3}{|p{9cm}|}{No aplica.} \\
    \hline \textit{Salida} & \multicolumn{3}{|p{9cm}|}{Se mostrar\'a la duracci\'on de la llamada realizada en el caso de que se realize una llamda. En otro caso, no se muestra nada m\'as.} \\ \hline
    \end{tabular}
\end{center}

\begin{center}
    \begin{tabular}{|p{2.6cm}|p{12cm}|}
    \hline
    \textbf{Identificador} & RF-12\\
    \hline
    \textbf{Descripcion} & Control de subida y bajada de persianas motorizadas con sistema de reporte del estado actual de las mismas\\
    \hline
    \textbf{Necesidad} & DESEABLE, OPCIONAL\\
    \hline
    \textbf{Dificultad} & ALTA\\
    \hline
    \textbf{Prioridad} & MEDIA, BAJA\\
    \hline
    \end{tabular}
\end{center}

\begin{center}
    \begin{tabular}{|p{2.6cm}|p{12cm}|}
    \hline
    \textbf{Identificador} & RF-13\\
    \hline
    \textbf{Descripcion} & Control de subida y bajada de toldos motorizados con sistema de reporte del estado actual de las mismas\\
    \hline
    \textbf{Necesidad} & DESEABLE, OPCIONAL\\
    \hline
    \textbf{Dificultad} & ALTA\\
    \hline
    \textbf{Prioridad} & MEDIA, BAJA\\
    \hline
    \end{tabular}
\end{center}

\begin{center}
    \begin{tabular}{|p{2.6cm}|p{12cm}|}
    \hline
    \textbf{Identificador} & RF-14\\
    \hline
    \textbf{Descripcion} & Control de sensores de movimiento\\
    \hline
    \textbf{Necesidad} & DESEABLE\\
    \hline
    \textbf{Dificultad} & MEDIA\\
    \hline
    \textbf{Prioridad} & BAJA\\
    \hline
    \end{tabular}
\end{center}

\begin{center}
    \begin{tabular}{|p{2.6cm}|p{12cm}|}
    \hline
    \textbf{Identificador} & RF-15\\
    \hline
    \textbf{Descripcion} & Gestión del historial de uso de dispositivos controlados por el sistema\\
    \hline
    \textbf{Necesidad} & DESEABLE, OPCIONAL\\
    \hline
    \textbf{Dificultad} & ALTA\\
    \hline
    \textbf{Prioridad} & MEDIA, BAJA\\
    \hline
    \end{tabular}
\end{center}

\begin{tabular}{|p{3cm}|p{4cm}|p{4cm}|p{4cm}|}
\hline \multicolumn{3}{|p{9cm}|}{\textit{Control de una regleta.}} & \textit{Versi\'on 1.0} \\
\hline \textit{Codigo} & \textit{Fecha} & \textit{Autor} & \textit{Grado de necesidad} \\
RF-16 & 10/12/2013 & Tirado, Colin & Alto \\
\hline \textit{Descripci\'on} & \multicolumn{3}{|p{9cm}|}{El usuario podr\'a apagar o encender dispositivos conectados a una regleta.} \\
\hline \textit{Entrada} & \multicolumn{3}{|p{9cm}|}{El usuario selecciona el bot\'on correspondiente situado en la zona donde se encuentra dicho elemento en el entorno EHC.} \\
\hline \textit{Precondici\'on} & \multicolumn{3}{|p{9cm}|}{
\begin{itemize}
\item El usuario est\'a identificado en el sistema EHC.
\item Tiene configurado un entorno EHC.
\item Tiene instalado y configurado una regleta para el entorno EHC.
\end{itemize}
} \\
\hline \textit{Acciones} & \multicolumn{3}{|p{9cm}|}{
\begin{enumerate}
\item El usuario se identifica en el sistema EHC.
\item Tras identificarse, acceder\'a a la secci\'on \textit{Gesti\'on} de la aplicaci\'on.
\item Selecciona la zona del entorno EHC donde desea controlar el elemento.
\item Realiza la acci\'on sobre el elemento a controlar.
\end{enumerate}
} \\
\hline \textit{Postcondici\'on} & \multicolumn{3}{|p{9cm}|}{Cambia el estado de  en el sistema EHC.} \\
\hline \textit{Salida} & \multicolumn{3}{|p{9cm}|}{Se realizar\'a la acci\'on enviada por el usuario.} \\ \hline
\end{tabular}
\begin{center}
    \begin{tabular}{|p{3cm}|p{4cm}|p{4cm}|p{4cm}|}
    \hline \multicolumn{3}{|p{9cm}|}{\textit{Mostrar texto en un display}} & \textit{Versi\'on 1.0} \\
    \hline \textit{Codigo} & \textit{Fecha} & \textit{Autor} & \textit{Grado de necesidad} \\
    RF-17 & 10/12/2013 & Tirado, Colin & Alto \\
    \hline \textit{Descripci\'on} & \multicolumn{3}{|p{9cm}|}{El usuario podr\'a dejar un mensaje en un display en el entorno EHC.} \\
    \hline \textit{Entrada} & \multicolumn{3}{|p{9cm}|}{El usuario selecciona el bot\'on correspondiente situado en la zona donde se encuentra dicho elemento en el entorno EHC.} \\
    \hline \textit{Precondici\'on} & \multicolumn{3}{|p{9cm}|}{
    \begin{itemize}
    \item El usuario est\'a identificado en el sistema EHC.
    \item Tiene configurado un entorno EHC.
    \item Tiene instalado y configurado un display en el entorno EHC.
    \end{itemize}
    } \\
    \hline \textit{Acciones} & \multicolumn{3}{|p{9cm}|}{
    \begin{enumerate}
    \item El usuario se identifica en el sistema EHC.
    \item Tras identificarse, acceder\'a a la secci\'on \textit{Gesti\'on} de la aplicaci\'on.
    \item Selecciona la zona del entorno EHC donde desea controlar el elemento.
    \item Selecciona el display.
    \item Introduce el texto y presiona el bot\'on \textit{Enviar}.
    \end{enumerate}
    } \\
    \hline \textit{Postcondici\'on} & \multicolumn{3}{|p{9cm}|}{Env\'ia el mensaje introducido al usuario al servidor.} \\
    \hline \textit{Salida} & \multicolumn{3}{|p{9cm}|}{Se realizar\'a la acci\'on enviada por el usuario.} \\ \hline
    \end{tabular}
\end{center}

\begin{center}
    \begin{tabular}{|p{3cm}|p{4cm}|p{4cm}|p{4cm}|}
    \hline \multicolumn{3}{|p{9cm}|}{\textit{Control del riego}} & \textit{Versi\'on 1.0} \\
    \hline \textit{Codigo} & \textit{Fecha} & \textit{Autor} & \textit{Grado de necesidad} \\
    RF-18 & 10/12/2013 & Tirado, Colin & Alto \\
    \hline \textit{Descripci\'on} & \multicolumn{3}{|p{9cm}|}{El usuario podr\'a controlar el riego instalado en el entorno EHC .} \\
    \hline \textit{Entrada} & \multicolumn{3}{|p{9cm}|}{El usuario selecciona el bot\'on correspondiente situado en la zona donde se encuentra dicho elemento en el entorno EHC.} \\
    \hline \textit{Precondici\'on} & \multicolumn{3}{|p{9cm}|}{
    \begin{itemize}
    \item El usuario est\'a identificado en el sistema EHC.
    \item Tiene configurado un entorno EHC.
    \item Tiene instalado y configurado un sistema de riego autom\'atico en el entorno EHC.
    \end{itemize}
    } \\
    \hline \textit{Acciones} & \multicolumn{3}{|p{9cm}|}{
    \begin{enumerate}
    \item El usuario se identifica en el sistema EHC.
    \item Tras identificarse, acceder\'a a la secci\'on \textit{Gesti\'on} de la aplicaci\'on.
    \item Selecciona la zona del entorno EHC donde desea controlar el elemento.
    \item Realiza la acci\'on sobre el elemento a controlar.
    \end{enumerate}
    } \\
    \hline \textit{Postcondici\'on} & \multicolumn{3}{|p{9cm}|}{Env\'ia la petici\'on solicitada por el usuario al servidor del sistema} \\
    \hline \textit{Salida} & \multicolumn{3}{|p{9cm}|}{Se realizar\'a la acci\'on enviada por el usuario.} \\ \hline
    \end{tabular}
\end{center}



\section{Requisitos no funcionales}
    \section{Requisitos relativos al producto}
    - Coste econ�mico de instalaci�n y mantenimiento del sistema reducido.\newline

- ?Buena? velocidad de respuesta del sistema. Sistema en ?tiempo real?.\newline

- Alto nivel de seguridad en todo el sistema. Control de autenticaci�n.\newline

- Sistema accesible, facil de usar\newline


\section{Requisitos externos}
    /*Para definir estos deberiamos estar muy puestos en el mundo de la domótica. Enterarnos de que estandares debemos cumplir (por ej: de seguridad, etc)\newline
Para hacer esto nos ayudara mucho ojear bien (detalladamente) el resto de proyectos de domótica similares que encontramos.*/\newline

- ¿ Cumplir algun estandar de seguridad ?\newline
/*Supongo que para poder comercializar algo tendran que darnos el visto bueno (legal) alguien*/\newline


\section{Requisitos de dominio}
    TODO

Puesto que a�n no tenemos muchas tablas en esto del mundo dom�tico no conocemos los requisitos de dominio.

Si alguien se entera de alguno que lo comunique.

