\chapter{Introducción}

    ***********************\newline
    TODO: Las secciones que estan recogidas en esta Introduccion son requisitos que hay que clasificar convenientemente en los siguientes capitulos (los que distuinguen los tipos de requisitos)
    ***********************\newline

\section{Características del usuario}
    Los usuarios de la aplicación deberán estar registrados para poder acceder a ella. El sistema contará con una base de datos donde estarán almacenados los nombres de usuario y contrase\~nas. Dichos usuarios podrán acceder a la aplicación y hacer uso de ella.

    \subsection{Perfil del usuario}
        Cada usuario tendrá un perfil específico para que su interacción con el sistema sea correcto y no conlleve a fallos:
        \begin{description}
            \item[Administrador del sistema] Usuario con gran conocimiento en el manejo del sistema con una previa capacitación por parte de la empresa. Encargado de manejar el sistema con gran responsabilidad sobre los criterios de permisos de los usuarios.

            \item[Usuario principal] Persona que interactuará continuamente con el sistema, su control sobre el mismo ha de ser eficiente. El conocimiento del manejo del sistema será adquirido por el usuario gracias a una peque\~na sesión de contacto guiada por el administrador, aunque la propia aplicación estará capacitada con una presentación/tutorial guiada/o del uso de la misma.

            \item[Visitante] Persona que interactuará ocasionalmente con el sistema, su control sobre el mismo ha de ser configurada por el usuario principal. El conocimiento del manejo del sistema será adquirido gracias a la presentación/tutorial guiada/o del uso de la aplicación.
        \end{description}

    \subsection{Jerarquía de usuarios}

\section{Restricciones}
    A continuación se presenta una lista de restricciones impuestas por los desarrolladores del sistema:

    \subsection{Limitaciones de usuario}
        \begin{description}
            \item[Administrador] Ninguna limitación.

            \item[Usuario principal] Tendrá la capacidad de crear perfiles de 'invitados' y configurar sus permisos, limitando así posibles accesos no deseados. Se verá mermado su control a la hora de a\~nadir y configurar más recursos al sistema.

            \item[Perfil de invitado] Su acceso a ciertas características del sistema se verá limitado debido a las posibles restricciones de su perfil.
        \end{description}

    \subsection{Políticas reguladoras}
        La aplicación se desarrollará mediante software de licencia abierta, por lo tanto no se deberá pagar por su uso. Asimismo, la utilización de la aplicación se hará mediante las políticas establecidas por este tipo de licenciamiento.

    \subsection{Limitaciones hardware}
        Para el completo funcionamiento de esta aplicación será necesario un servidor y una serie de mecanismos hardware que se instalarán en el lugar donde se quiera implantar el sistema EHC.

    \subsection{Interfaces con otras aplicaciones}
        Debido a que el sistema no interactúa con otros sistemas y es autónomo, no se desarrollarán interfaces con otras aplicaciones. Las conexiones necesarias para la utilización del servidor y los mecanismos hardware, se harán por medio de la configuración de los mismos.


    \subsection{Funciones de control}
        El sistema EHC administrará y controlará los permisos que tiene cada usuario para que su accesibilidad se haga de una manera correcta, de tal forma que acceda a la información que le corresponde de acuerdo a sus permisos. También deberá tener controles adecuados para la gestión y validación de datos, así como para las tareas, mandatos e instrucciones pedidos al sistema.

        Por otro lado, a cargo de varios supervisores estarán encargadas las tareas de revisión de elementos como el c\'digo, el repositorio, fechas de entrega, documentación, etc.

    \subsection{Requisitos del lenguaje}
        Tanto el material que vaya dirigido al usuario como la aplicación deben estar en espa\~nol.

    \subsection{Lenguajes de programación}
        Se hará uso de las tecnologías; Java (Android) y Objective-c (iOS).

    \subsection{Requisitos de fiabilidad}
        Previamente, para evitar posibles futuros fallos, la aplicación será sometida a multitud de pruebas y tests garantizando que se entregará al usuario un programa totalmente funcional y eficiente.

    \subsection{Suposiciones y dependencias}
        Se han supuesto los requisitos para que el sistema funcione correctamente con especificaciones estándar.
        Es viable tratar de ampliar la capacidad de mercado capacitando al sistema con versiones de la aplicación en diferentes idiomas.

    \subsection{Requisitos futuros}
        En un futuro se tratarán de implementar nuevas funcionalidades que requerirán una modificación de los requisitos iniciales:
