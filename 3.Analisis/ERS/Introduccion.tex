\chapter{Introducción}

    En este documento se expondran los requisitos funcionales (funcionalidades) y no funcoinales (restricciones) que el sistema desarrollado debe 

\section{Tipos de Participantes}
    Todos los participantes que interactuen con el sistema perteneceran a alguno de los siguientes tipos de participante:
    
    \begin{description}
        \item[Administrador] Personal cualificado con conocimientos avanzados sobre el funcionamiento del sistema encargado realizar tareas de instalacion, mantenimiento y reparación del mismo. Este tipo de participante tendra acceso total al sistema sin ningun tipo de restriccion.

        \item[Superusuario] Usuario encargado de gestionar las cuentas de usuario. Este tipo de participante dispondra de privilegios especiales para crear o eliminar cuentas de usuario, y otorgar o revocar el acceso de los usuarios a determinadas funcionalidades del sistema.

        \item[Usuario] Usuario final del sistema encargado de hacer uso responsable de las funcionalidades del sistema a las que se le haya dado acceso por parte de un Administrador o Superusuario.
    \end{description}
    
    
    
 
