\chapter{Requisitos de sistema}

    Existen muchas clases de requisitos, pero en esta seccion nos vamos a centrar unicamente en los que mas nos interesan, los requisitos de sistema, los cuales subdividiremos en:
    \begin{description}
        \item[Requisitos funcionales o funcionalidades:]
            Definen una funciónalidad del sistema, es decir, describen lo que el sistema desarrollado debe ser capaz de hacer.
        \item[Requisitos no funcionales o restricciones:]
            Definen restricciones a las que debe acogerse el sistema desarrollado.
  
    \end{description}
  

\section{Requisitos funcionales}
    \begin{center}
    \begin{tabular}{|p{2.6cm}|p{12cm}|}
    \hline
    \textbf{Identificador} & RF-01\\
    \hline
    \textbf{Descripcion} & Control de TV modelo X mediante codigos del mando modelo Y. Se ofrecera una funcionalidad basica (parcial) del mando a distancia.\\
    \hline
    \textbf{Necesidad} & DESEABLE\\
    \hline
    \textbf{Dificultad} & BAJA\\
    \hline
    \textbf{Prioridad} & ALTA\\
    \hline
    \end{tabular}
\end{center}

\begin{center}
    \begin{tabular}{|p{2.6cm}|p{12cm}|}
    \hline
    \textbf{Identificador} & RF-02\\
    \hline
    \textbf{Descripcion} & Captacion de informacion relativa a la programación ofrecida en la TV modelo X\\
    \hline
    \textbf{Necesidad} & DESEABLE\\
    \hline
    \textbf{Dificultad} & BAJA\\
    \hline
    \textbf{Prioridad} & MEDIA\\
    \hline
    \end{tabular}
\end{center}

\begin{center}
    \begin{tabular}{|p{2.6cm}|p{12cm}|}
    \hline
    \textbf{Identificador} & RF-03\\
    \hline
    \textbf{Descripcion} & Captación de información relativa a sensores meteorológicos (meteorología: lluvia, viento, sol, etc)\\
    \hline
    \textbf{Necesidad} & DESEABLE\\
    \hline
    \textbf{Dificultad} & MEDIA\\
    \hline
    \textbf{Prioridad} & BAJA\\
    \hline
    \end{tabular}
\end{center}

\begin{center}
    \begin{tabular}{|p{2.6cm}|p{12cm}|}
    \hline
    \textbf{Identificador} & RF-04\\
    \hline
    \textbf{Descripcion} & Control por voz de parte del sistema (Falta por especificar que vamos a controlar por voz exactamente)\\
    \hline
    \textbf{Necesidad} & OPCIONAL\\
    \hline
    \textbf{Dificultad} & ALTA\\
    \hline
    \textbf{Prioridad} & BAJA\\
    \hline
    \end{tabular}
\end{center}

\begin{center}
    \begin{tabular}{|p{2.6cm}|p{12cm}|}
    \hline
    \textbf{Identificador} & RF-05\\
    \hline
    \textbf{Descripcion} & Control del telefonillo modelo X\\
    \hline
    \textbf{Necesidad} & DESEABLE\\
    \hline
    \textbf{Dificultad} & ALTA\\
    \hline
    \textbf{Prioridad} & MEDIA\\
    \hline
    \end{tabular}
\end{center}

\begin{center}
    \begin{tabular}{|p{3cm}|p{4cm}|p{4cm}|p{4cm}|}
    \hline \multicolumn{3}{|p{9cm}|}{\textit{A\~nadir un evento}} & \textit{Versi\'on 1.0} \\
    \hline \textit{Codigo} & \textit{Fecha} & \textit{Autor} & \textit{Grado de necesidad} \\
    RF-06 & 10/12/2013 & Tirado, Colin & Alto \\
    \hline \textit{Descripci\'on} & \multicolumn{3}{|p{9cm}|}{El usuario podr\'a a\~nadir un evento a su entorno EHC} \\
    \hline \textit{Entrada} & \multicolumn{3}{|p{9cm}|}{El usuario selecciona el bot\'on de \textit{Eventos} situado en la ventana principal de la aplicaci\'on.} \\
    \hline \textit{Precondici\'on} & \multicolumn{3}{|p{9cm}|}{El usuario est\'a identificado en el sistema EHC.} \\
    \hline \textit{Acciones} & \multicolumn{3}{|p{9cm}|}{
    \begin{enumerate}
    \item El usuario se identifica en el sistema EHC.
    \item Tras identificarse, acceder\'a a la secci\'on \textit{Eventos} de la aplicaci\'on
    \item Selecciona el bot\'on \textit{A\~nadir evento}.
    \end{enumerate}
    } \\
    \hline \textit{Postcondici\'on} & \multicolumn{3}{|p{9cm}|}{Se a\~nade un nuevo evento asociado al entorno EHC correspondiente.} \\
    \hline \textit{Salida} & \multicolumn{3}{|p{9cm}|}{Se notificar\'a al usuario si se ha a\~nadido correctamente el evento.} \\ \hline
    \end{tabular}
\end{center}

\begin{center}
    \begin{tabular}{|p{2.6cm}|p{12cm}|}
    \hline
    \textbf{Identificador} & RF-07\\
    \hline
    \textbf{Descripcion} & Captación de informacion web relativa a la prevision meteorológica de la zona elegida por el usuario\\
    \hline
    \textbf{Necesidad} & DESEABLE, OPCIONAL\\
    \hline
    \textbf{Dificultad} & ALTA\\
    \hline
    \textbf{Prioridad} & MEDIA, BAJA\\
    \hline
    \end{tabular}
\end{center}

\begin{center}
    \begin{tabular}{|p{2.6cm}|p{12cm}|}
    \hline
    \textbf{Identificador} & RF-08\\
    \hline
    \textbf{Descripcion} & Captación de infomración relativa al consumo energético en tiempo real (factura de la luz)
    \\
    \hline
    \textbf{Necesidad} & DESEABLE, OPCIONAL\\
    \hline
    \textbf{Dificultad} & ALTA\\
    \hline
    \textbf{Prioridad} & MEDIA, BAJA\\
    \hline
    \end{tabular}
\end{center}

\begin{center}
    \begin{tabular}{|p{3cm}|p{4cm}|p{4cm}|p{4cm}|}
    \hline \multicolumn{3}{|p{9cm}|}{\textit{Controlar una trampilla}} & \textit{Versi\'on 1.0} \\
    \hline \textit{Codigo} & \textit{Fecha} & \textit{Autor} & \textit{Grado de necesidad} \\
    RF-09 & 10/12/2013 & Tirado, Colin & Medio \\
    \hline \textit{Descripci\'on} & \multicolumn{3}{|p{9cm}|}{El usuario podr\'a controlar desde su aplicaci\'on EHC una trampilla instalada en el entorno EHC del usuario.} \\
    \hline \textit{Entrada} & \multicolumn{3}{|p{9cm}|}{El usuario selecciona el bot\'on correspondiente situado en la zona donde se encuentra dicho elemento en el entorno EHC.} \\
    \hline \textit{Precondici\'on} & \multicolumn{3}{|p{9cm}|}{
    \begin{itemize}
    \item El usuario est\'a identificado en el sistema EHC
    \item Tiene configurado un entorno EHC.
    \item Tiene instalado una trampilla en el entorno EHC
    \end{itemize}
    } \\
    \hline \textit{Acciones} & \multicolumn{3}{|p{9cm}|}{
    \begin{enumerate}
    \item El usuario se identifica en el sistema EHC.
    \item Tras identificarse, acceder\'a a la secci\'on \textit{Gesti\'on} de la aplicaci\'on.
    \item Selecciona la zona del entorno EHC donde desea controlar el elemento.
    \item Realiza la acci\'on sobre el elemento a controlar.
    \end{enumerate}
    } \\
    \hline \textit{Postcondici\'on} & \multicolumn{3}{|p{9cm}|}{Cambia el estado de la trampilla en el sistema EHC.} \\
    \hline \textit{Salida} & \multicolumn{3}{|p{9cm}|}{Se realizar\'a la acci\'on enviada por el usuario.} \\ \hline
    \end{tabular}
\end{center}

\begin{center}
    \begin{tabular}{|p{2.6cm}|p{12cm}|}
    \hline
    \textbf{Identificador} & RF-06\\
    \hline
    \textbf{Descripcion} & Control de subida y bajada de la puerta del garaje con sistema de reporte del estado actual de las misma\\
    \hline
    \textbf{Necesidad} & DESEABLE\\
    \hline
    \textbf{Dificultad} & ALTA\\
    \hline
    \textbf{Prioridad} & BAJA\\
    \hline
    \end{tabular}
\end{center}

\begin{center}
    \begin{tabular}{|p{2.6cm}|p{12cm}|}
    \hline
    \textbf{Identificador} & RF-06\\
    \hline
    \textbf{Descripcion} & Control del telefonillo modelo X\\
    \hline
    \textbf{Necesidad} & DESEABLE, OPCIONAL\\
    \hline
    \textbf{Dificultad} & ALTA\\
    \hline
    \textbf{Prioridad} & MEDIA, BAJA\\
    \hline
    \end{tabular}
\end{center}

\begin{center}
    \begin{tabular}{|p{2.6cm}|p{12cm}|}
    \hline
    \textbf{Identificador} & RF-06\\
    \hline
    \textbf{Descripcion} & Control del telefonillo modelo X\\
    \hline
    \textbf{Necesidad} & DESEABLE, OPCIONAL\\
    \hline
    \textbf{Dificultad} & ALTA\\
    \hline
    \textbf{Prioridad} & MEDIA, BAJA\\
    \hline
    \end{tabular}
\end{center}

\begin{center}
    \begin{tabular}{|p{2.6cm}|p{12cm}|}
    \hline
    \textbf{Identificador} & RF-13\\
    \hline
    \textbf{Descripcion} & Control de subida y bajada de toldos motorizados con sistema de reporte del estado actual de las mismas\\
    \hline
    \textbf{Necesidad} & DESEABLE\\
    \hline
    \textbf{Dificultad} & MEDIA\\
    \hline
    \textbf{Prioridad} & BAJA\\
    \hline
    \end{tabular}
\end{center}

\begin{center}
    \begin{tabular}{|p{2.6cm}|p{12cm}|}
    \hline
    \textbf{Identificador} & RF-06\\
    \hline
    \textbf{Descripcion} & Control del telefonillo modelo X\\
    \hline
    \textbf{Necesidad} & DESEABLE, OPCIONAL\\
    \hline
    \textbf{Dificultad} & ALTA\\
    \hline
    \textbf{Prioridad} & MEDIA, BAJA\\
    \hline
    \end{tabular}
\end{center}

\begin{center}
    \begin{tabular}{|p{2.6cm}|p{12cm}|}
    \hline
    \textbf{Identificador} & RF-06\\
    \hline
    \textbf{Descripcion} & Control del telefonillo modelo X\\
    \hline
    \textbf{Necesidad} & DESEABLE, OPCIONAL\\
    \hline
    \textbf{Dificultad} & ALTA\\
    \hline
    \textbf{Prioridad} & MEDIA, BAJA\\
    \hline
    \end{tabular}
\end{center}

\begin{center}
    \begin{tabular}{|p{2.6cm}|p{12cm}|}
    \hline
    \textbf{Identificador} & RF-06\\
    \hline
    \textbf{Descripcion} & Control del telefonillo modelo X\\
    \hline
    \textbf{Necesidad} & DESEABLE, OPCIONAL\\
    \hline
    \textbf{Dificultad} & ALTA\\
    \hline
    \textbf{Prioridad} & MEDIA, BAJA\\
    \hline
    \end{tabular}
\end{center}

\begin{center}
    \begin{tabular}{|p{2.6cm}|p{12cm}|}
    \hline
    \textbf{Identificador} & RF-06\\
    \hline
    \textbf{Descripcion} & Control del telefonillo modelo X\\
    \hline
    \textbf{Necesidad} & DESEABLE, OPCIONAL\\
    \hline
    \textbf{Dificultad} & ALTA\\
    \hline
    \textbf{Prioridad} & MEDIA, BAJA\\
    \hline
    \end{tabular}
\end{center}

\begin{center}
    \begin{tabular}{|p{2.6cm}|p{12cm}|}
    \hline
    \textbf{Identificador} & RF-06\\
    \hline
    \textbf{Descripcion} & Control del telefonillo modelo X\\
    \hline
    \textbf{Necesidad} & DESEABLE, OPCIONAL\\
    \hline
    \textbf{Dificultad} & ALTA\\
    \hline
    \textbf{Prioridad} & MEDIA, BAJA\\
    \hline
    \end{tabular}
\end{center}



\section{Requisitos no funcionales}
    \subsection{Desempeño}
Garantizar la confiabilidad, la seguridad y el desempe\~no de la aplicaci\'on a los diferentes usuarios en el entorno EHC. En este sentido el control y la gesti\'on del entorno EHC ser\'a totalmente funcional, garantizando que el tiempo de respuesta sea casi imperceptible.
La aplicaci\'on debe estar en capacidad de dar respuesta al acceso y a los procesos creados por los mandatos de los usuarios con tiempos de respuesta aceptables y uniformes, en la medida de las posibilidades tecnol\'ogicas del entorno EHC (servidor, mecanismos hardware, conexi\'on a internet), en per\'iodos de alta, media y baja demanda de uso del sistema.

\subsection{Extensibilidad}
La aplicaci\'on debe ser construida sobre una arquitectura de desarrollo evolutivo e incremental, de manera tal que nuevas funcionalidades y requerimientos relacionados puedan ser incorporados afectando el c\'odigo existente de la menor manera posible; para ello deben incorporarse aspectos de reutilizaci\'on de componentes. As\'i mismo, debe estar en capacidad de permitir en el futuro el desarrollo de nuevas funcionalidades, modificar o eliminar funcionalidades despu\'es de su construcci\'on y puesta en marcha inicial.

\subsection{Disponibilidad}
Estar disponible 100\% o muy cercano a esta disponibilidad las 24 horas del d\'ia. Dependencia por parte de la conexi\'on a internet y/o posible mantenimiento del servidor cuando el usuario est\'e fuera del entorno EHC.
\subsection{Facilidad de uso}
La aplicacion debe ser de f\'acil uso y entrenamiento por parte de los usuarios. Gracias a navegaci\'on a trav\'es de interfaces intuitivas, sencillas y amigables y capacitaci\'on en el uso de la aplicaci\'on gracias a una herramienta gu\'ia incluida en la propia aplicaci\'on.
Tambi\'en deber\'a presentar mensajes de error que permitan al usuario identificar el tipo de error y comunicarse con el administrador del sistema.

\subsection{Facilidad para las pruebas}
La aplicaci\'on debe contar con facilidades para la identificaci\'on de la localizaci\'on de los errores durante la etapa de pruebas y de operaci\'on posterior.

\subsection{Instalaci\'on}
La aplicaci\'on debe ser f\'acil de instalar en dispositivos que cuenten con sistemas operativos iOS y Android.
\subsection{Mantenibilidad}
La aplicación deber\'a estar completamente documentada, cada uno de los componentes software que forman parte de la solución propuesta deber\'an estar debidamente documentados tanto en el c\'odigo fuente como en los manuales de administraci\'on y de usuario.
Contar\'a con una interfaz de administraci\'on que incluya: Administraci\'on de usuarios, Creaci\'on, administraci\'on y gesti\'on de recursos. En cada una de \'estas secciones deber\'a ofrecer todas las opciones de administraci\'on disponibles para el control de las mismas.
Por otro lado, debe estar en capacidad de permitir en el futuro su f\'acil mantenimiento con respecto a los posibles errores que se puedan presentar durante la operación del sistema.

\subsection{Operatividad}
La aplicaci\'on deber\'a poder ser administrada remotamente por las personas encargadas o designadas por la empresa EHC.
\subsection{Seguridad}
El acceso al Sistema debe estar restringido por el uso de claves asignadas a cada uno de los usuarios. S\'olo podr\'an ingresar al Sistema las personas que est\'en registradas, estos usuarios ser\'an clasificados en varios tipos de usuarios (o roles) seg\'un los permisos que tengan asignados.
Respecto a la confidencialidad, la aplicaci\'on debe estar en capacidad de rechazar accesos y modificaciones indebidas (no autorizadas) y proveer los servicios requeridos por los usuarios leg\'itimos del sistema. Adem\'as deber\'a contar con mecanismos que permitan el registro de actividades con identificaci\'on de los usuarios que los realizaron.
La aplicaci\'on debe contar con pistas de auditor\'ia de las actividades que se realizan sobre el sistema con niveles razonables para su reconstrucci\'on e identificaci\'on de los hechos.

\subsection{Interoperabilidad}
La aplicaci\'on debe estar en capacidad de interactuar con los otros componentes del sistema EHC (servidor y mecanismos hardware).





%\section{Requisitos de dominio}
%    TODO

Puesto que a�n no tenemos muchas tablas en esto del mundo dom�tico no conocemos los requisitos de dominio.

Si alguien se entera de alguno que lo comunique.

