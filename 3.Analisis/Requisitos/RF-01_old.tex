\begin{center}
    \begin{tabular}{|p{3cm}|p{4cm}|p{4cm}|p{4cm}|}
        \hline \multicolumn{3}{|p{9cm}|}{\textit{Acceso a la aplicaci\'on}} & \textit{Versi\'on 1.0} \\
        \hline \textit{Codigo} & \textit{Fecha} & \textit{Autor} & \textit{Grado de necesidad} \\
        RF-01 & 09/12/2013 & Tirado, Colin & Alto \\
        \hline \textit{Descripci\'on} & \multicolumn{3}{|p{9cm}|}{Identificaci\'on del usuario en el sistema.} \\
        \hline \textit{Entrada} & \multicolumn{3}{|p{9cm}|}{Nombre de usuario y contrase\~na del usuario.} \\
        \hline \textit{Precondici\'on} & \multicolumn{3}{|p{9cm}|}{Conexi\'on a la red.} \\
        \hline \textit{Acciones} & \multicolumn{3}{|p{9cm}|}{
        \begin{enumerate}
            \item Acceder a la aplicaci\'on.
            \item Introducir el nombre de usuario en el campo \textit{Usuario}.
            \item Introducir la contrase\~na en el campo \textit{contrase\~na}.
            \item Presionar el bot\'on \textit{Acceder}.
        \end{enumerate}
        } \\
        \hline \textit{Postcondici\'on} & \multicolumn{3}{|p{9cm}|}{Se contempla dos situaciones:
        \begin{enumerate}
            \item Si se identifica con \'exito: se volcar\'a la informaci\'on correspondiente del usuario.
            \item En caso contrario: no se aplica.
        \end{enumerate}
        } \\
        \hline \textit{Salida} & \multicolumn{3}{|p{9cm}|}{Se contemplan tres situaciones
        \begin{enumerate}
            \item Si el usuario no existe: se notificar\'a que el usuario intruducido no est\'a registrado en el sistema.
            \item Si se ha introducido de forma err\'onea la contrase\~na: se notificar\'a que se ha introducido mal la contrase\~na.
            \item Si se introduce el usuario y contrase\~na correctamente: se mostrar\'a la ventana principal de la aplicaci\'on.
        \end{enumerate} } \\ \hline
    \end{tabular}
\end{center}
