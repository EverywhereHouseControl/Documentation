\begin{center}
    \begin{tabular}{|p{3cm}|p{4cm}|p{4cm}|p{4cm}|}
    \hline \multicolumn{3}{|p{9cm}|}{\textit{Control de una regleta.}} & \textit{Versi\'on 1.0} \\
    \hline \textit{Codigo} & \textit{Fecha} & \textit{Autor} & \textit{Grado de necesidad} \\
    RF-16 & 10/12/2013 & Tirado, Colin & Alto \\
    \hline \textit{Descripci\'on} & \multicolumn{3}{|p{9cm}|}{El usuario podr\'a apagar o encender dispositivos conectados a una regleta.} \\
    \hline \textit{Entrada} & \multicolumn{3}{|p{9cm}|}{El usuario selecciona el bot\'on correspondiente situado en la zona donde se encuentra dicho elemento en el entorno EHC.} \\
    \hline \textit{Precondici\'on} & \multicolumn{3}{|p{9cm}|}{
    \begin{itemize}
    \item El usuario est\'a identificado en el sistema EHC.
    \item Tiene configurado un entorno EHC.
    \item Tiene instalado y configurado una regleta para el entorno EHC.
    \end{itemize}
    } \\
    \hline \textit{Acciones} & \multicolumn{3}{|p{9cm}|}{
    \begin{enumerate}
    \item El usuario se identifica en el sistema EHC.
    \item Tras identificarse, acceder\'a a la secci\'on \textit{Gesti\'on} de la aplicaci\'on.
    \item Selecciona la zona del entorno EHC donde desea controlar el elemento.
    \item Realiza la acci\'on sobre el elemento a controlar.
    \end{enumerate}
    } \\
    \hline \textit{Postcondici\'on} & \multicolumn{3}{|p{9cm}|}{Cambia el estado de  en el sistema EHC.} \\
    \hline \textit{Salida} & \multicolumn{3}{|p{9cm}|}{Se realizar\'a la acci\'on enviada por el usuario.} \\ \hline
    \end{tabular}
\end{center}
