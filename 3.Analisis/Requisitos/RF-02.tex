\begin{center}
    \begin{tabular}{|p{3cm}|p{4cm}|p{4cm}|p{4cm}|}
    \hline \multicolumn{3}{|p{9cm}|}{\textit{Control de luces}} & \textit{Versi\'on 1.0} \\
    \hline \textit{Codigo} & \textit{Fecha} & \textit{Autor} & \textit{Grado de necesidad} \\
    RF-02 & 12/12/2013 & Tirado, Colin & Alto \\
    \hline \textit{Descripci\'on} & \multicolumn{3}{|p{9cm}|}{El usuario podr\'a encender, apagar o regular la intensidad de un aparato de iluminaci\'on} \\
    \hline \textit{Entrada} & \multicolumn{3}{|p{9cm}|}{Un valor entre 0 y 100 indicando la intensidad de la luz deseada por el usuario, siendo 0 apagado y 100 encendido con la máxima intensidad } \\
    \hline \textit{Precondic\'ion} & \multicolumn{3}{|p{9cm}|}{Conexi\'on establecida en la red.} \\
    \hline \textit{Acciones} & \multicolumn{3}{|p{9cm}|}{
    \begin{enumerate}
    \item El usuario se identifica en el sistema EHC.
    \item Tras identificarse, acceder\'a a la secci\'on \textit{Gesti\'on} de la aplicaci\'on.
    \item Selecciona la zona del entorno EHC donde desea controlar el elemento.
    \item Realiza la acci\'on sobre el elemento a controlar.
    \end{enumerate}
    } \\
    \hline \textit{Postcondici\'on} & \multicolumn{3}{|p{9cm}|}{No se aplica.} \\
    \hline \textit{Salida} & \multicolumn{3}{|p{9cm}|}{Se realizar\'a la acci\'on enviada por el usuario.} \\ \hline
    \end{tabular}
\end{center}
