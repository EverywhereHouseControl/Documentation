\chapter{Descripción de Stakeholders y usuarios}
    Para proveer de una forma efectiva el servicio y que se ajuste a las necesidades de los usuarios, es necesario identificar e involucrar a todos los participantes en el proyecto como parte del proceso de modelado de requerimientos. También es necesario identificar a los usuarios del sistema y asegurarse de que el conjunto de participantes en el proyecto los representa adecuadamente. Esta sección muestra un perfil de los participantes y de los usuarios involucrados en el proyecto, así como los problemas más importantes que éstos perciben para enfocar la solución propuesta hacia ellos. No describe sus requisitos específicos pero proporciona la justificación de por qué estos requisitos son necesarios.

\section{Resumen de \textit{Stakeholders}}
    \begin{tabular}{|p{4cm}|p{3cm}|p{10cm}|}
        \hline \textbf{Nombre} &  \textbf{Descripción} & \textbf{Responsbilidades} \\
        \hline Morales Lozano, Álvaro & Representante Global de la empresa EHC. &
        \begin{itemize}
            \item Seguimiento del desarrollo del proyecto a nivel global.
            \item Aprueba requisitos y funcionalidades a nivel global.
        \end{itemize} \\
        \hline Vicente Sánchez, Victor & Representante del Departamento de Software. &
        \begin{itemize}
            \item Seguimiento del desarrollo del proyecto a nivel software.
            \item Aprueba requisitos y funcionalidades.
        \end{itemize} \\
        \hline Gutiérrez, Héctor & Representante del Departamento de Hardware. &
        \begin{itemize}
            \item Seguimiento del desarrollo del proyecto a nivel hardware.
            \item Aprueba requisitos y funcionalidades.
        \end{itemize} \\
        \hline Ladrón De Guevara, Alejandro & Representante del Departamento de Servidor.&
        \begin{itemize}
            \item Seguimiento del desarrollo del proyecto a nivel servidor.
            \item Aprueba requisitos y funcionalidades.
        \end{itemize} \\
        \hline
    \end{tabular}

\section{Resumen de usuarios}
    \begin{tabular}{|p{3cm}|p{7cm}|p{6cm}|}
        \hline \textbf{Nombre} &  \textbf{Descripción} & \textbf{StakeHolder} \\
        \hline Administrador & Encargado de la instalación del sistema EHC en el cliente. Se encargará de  instalar/configurar el sistema en el entorno EHC. & Gutiérrez, Hector \\
        \hline Usuario & Participante del entorno EHC &   Ladrón De Guevara, Alejandro \par Vicente Sánchez, Victor \\
        \hline
    \end{tabular}

\section{Entorno de usuarios}
    Los usuarios entrarán al sistema identificándose sobre cualquier dispositivo móvil que tenga instalada la aplicación de gestión del sistema EHC o a través de la aplicación web. Tras este paso podrán controlar multitud de tareas y acciones que ofrecen los distintos recursos del entorno EHC navegando a través de interfaces ágiles y amigables.

\section{Perfil de \textit{Stakeholders}}
    \subsection{Departamento de Software}
        \begin{tabular}{|p{4cm}|p{12cm}|}
            \hline \textbf{Representante} & Vicente Sánchez, Victor. \\
            \hline \textbf{Descripción} & Representante del Departamento de Software. \\
            \hline \textbf{Tipo} &  [Experto] Estudiante de sistemas software. \\
            \hline \textbf{Responsabilidades} &  Lleva a cabo un seguimiento del desarrollo del proyecto y aprobación de  los requisitos y funcionalidades del sistema a nivel software.\\
            \hline \textbf{Criterio de éxito} &  [A definir por el cliente]\\
            \hline \textbf{Grado de participación} &  Revisión de requerimientos, desarrollo y estructura del sistema a nivel software.\\
            \hline \textbf{Comentarios} &  Ninguno\\
            \hline
        \end{tabular}


        \begin{tabular}{|p{4cm}|p{12cm}|}
            \hline \textbf{Representante} &  Maldonado, Miguel Alexander \\
            \hline \textbf{Descripción} &  \\
            \hline \textbf{Tipo} &  \\
            \hline \textbf{Responsabilidades} &  \\
            \hline \textbf{Criterio de éxito} &  \\
            \hline \textbf{Grado de participación} &  \\
            \hline \textbf{Comentarios} &  \\
            \hline
        \end{tabular}

    \subsection{Departamento de Hardware}

        \begin{tabular}{|p{4cm}|p{12cm}|}
            \hline \textbf{Representante} & Morales Lozano, Álvaro \\
            \hline \textbf{Descripción} &  Representante Global de la Empresa EHC\\
            \hline \textbf{Tipo} &  [Experto] Estudiante de gestión de sistemas.\\
            \hline \textbf{Responsabilidades} &  Encargado de mostrar las necesidades de cada usuario del sistema. Además, lleva a cabo un seguimiento del desarrollo del proyecto y aprobación de  los requisitos y funcionalidades del sistema a nivel global.\\
            \hline \textbf{Criterio de éxito} &  [A definir por el cliente]\\
            \hline \textbf{Grado de participación} &  Revisión de requerimientos, desarrollo y estructura del sistema a nivel global.\\
            \hline \textbf{Comentarios} &  Ninguno.\\
            \hline
        \end{tabular}

        \begin{tabular}{|p{4cm}|p{12cm}|}
            \hline \textbf{Representante} &  Gutiérrez, Héctor\\
            \hline \textbf{Descripción} &  Representante del Departamento de Hardware. \\
            \hline \textbf{Tipo} &  [Experto] Estudiante de sistemas hardware.\\
            \hline \textbf{Responsabilidades} &  Lleva a cabo un seguimiento del desarrollo del proyecto y aprobación de  los requisitos y funcionalidades del sistema a nivel hardware. [A definir por el cliente]\\
            \hline \textbf{Criterio de éxito} &  [A definir por el cliente]\\
            \hline \textbf{Grado de participación} &  Revisión de requerimientos, desarrollo y estructura del sistema a nivel hardware.\\
            \hline \textbf{Comentarios} &  Ninguno\\
            \hline
        \end{tabular}

        \begin{tabular}{|p{4cm}|p{12cm}|}
            \hline \textbf{Representante} & Guzman, Fernando \\
            \hline \textbf{Descripción} &  \\
            \hline \textbf{Tipo} &  \\
            \hline \textbf{Responsabilidades} &  \\
            \hline \textbf{Criterio de éxito} &  \\
            \hline \textbf{Grado de participación} &  \\
            \hline \textbf{Comentarios} &  \\
            \hline
        \end{tabular}

    \subsection{Departamento de Servidor}
        \begin{tabular}{|p{4cm}|p{12cm}|}
            \hline \textbf{Representante} &  Ladrón De Guevara, Alejandro\\
            \hline \textbf{Descripción} & Representante del Departamento de Servidor. \\
            \hline \textbf{Tipo} &  [Experto] Estudiante de sistemas con servidores. \\
            \hline \textbf{Responsabilidades} & Lleva a cabo un seguimiento del desarrollo del proyecto y aprobación de  los requisitos y funcionalidades del sistema a nivel del servidor. \\
            \hline \textbf{Criterio de éxito} & [A definir por el cliente] \\
            \hline \textbf{Grado de participación} & Revisión de requerimientos, desarrollo y estructura del sistema a nivel del servidor. \\
            \hline \textbf{Comentarios} &  Ninguno. \\
            \hline
        \end{tabular}

        \begin{tabular}{|p{4cm}|p{12cm}|}
            \hline \textbf{Representante} & Rey, José Antonio \\
            \hline \textbf{Descripción} &  \\
            \hline \textbf{Tipo} &  \\
            \hline \textbf{Responsabilidades} &  \\
            \hline \textbf{Criterio de éxito} &  \\
            \hline \textbf{Grado de participación} &  \\
            \hline \textbf{Comentarios} &  \\
            \hline
        \end{tabular}

        \begin{tabular}{|p{4cm}|p{12cm}|}
            \hline \textbf{Representante} & Saavedra, Luis Antonio \\
            \hline \textbf{Descripción} &  \\
            \hline \textbf{Tipo} &  \\
            \hline \textbf{Responsabilidades} &  \\
            \hline \textbf{Criterio de éxito} &  \\
            \hline \textbf{Grado de participación} &  \\
            \hline \textbf{Comentarios} &  \\
            \hline
        \end{tabular}


    \subsection{Departamento de documentación}

        \begin{tabular}{|p{4cm}|p{12cm}|}
            \hline \textbf{Representante} & Ochoa, Victor \\
            \hline \textbf{Descripción} &  \\
            \hline \textbf{Tipo} &  \\
            \hline \textbf{Responsabilidades} &  \\
            \hline \textbf{Criterio de éxito} &  \\
            \hline \textbf{Grado de participación} &  \\
            \hline \textbf{Comentarios} &  \\
            \hline
        \end{tabular}


        \begin{tabular}{|p{4cm}|p{12cm}|}
            \hline \textbf{Representante} & Tirado, Colin \\
            \hline \textbf{Descripción} &  \\
            \hline \textbf{Tipo} &  \\
            \hline \textbf{Responsabilidades} &  \\
            \hline \textbf{Criterio de éxito} &  \\
            \hline \textbf{Grado de participación} &  \\
            \hline \textbf{Comentarios} &  \\
            \hline
        \end{tabular}

\section{Tipos de usuario}

    \subsection{Administrador}
        \begin{tabular}{|p{4cm}|p{12cm}|}
            \hline \textbf{Representante} & Administrador \\
            \hline \textbf{Descripción} & Participante directo de la empresa EHC que se encarga de la instalación y el aprendizaje inicial del sistema EHC en el entorno instalado. \\
            \hline \textbf{Tipo} & Usuario casual del sistema. \\
            \hline \textbf{Responsabilidades} & Responsable de la instalación y del funcionamiento del sistema. \\
            \hline \textbf{Criterio de éxito} & [A definir por el cliente] \\
            \hline \textbf{Grado de participación} & Instalación del sistema en el lugar contratado. \par
            Mostrar y aconsejar al usuario como usar el sistema. \\
            \hline \textbf{Comentarios} & Ninguno. \\
            \hline
        \end{tabular}

    \subsection{Cliente}
        \begin{tabular}{|p{4cm}|p{12cm}|}
            \hline \textbf{Representante} & Cliente. \\
            \hline \textbf{Descripción} & Usuario (o empresa) que desee  tener de manera centralizada el control de los recursos del hogar (o centro de trabajo). \\
            \hline \textbf{Tipo} & Usuario casual del sistema. \\
            \hline \textbf{Responsabilidades} & Uso correcto de la instalación del sistema EHC \\
            \hline \textbf{Criterio de éxito} & [A definir por el cliente] \\
            \hline \textbf{Grado de participación} & [A definir por el cliente] \\
            \hline \textbf{Comentarios} & Ninguno. \\
            \hline
        \end{tabular}

\section{Alternativas y competencias}
    A continuación, se mostrará una lista de las principales alternativas al sistema
    %Pendiente de formatear
    \newline

    + OpenDomotica - https://opendomotica.wordpress.com/ [Este no se sirve de Android]
    \newline
        Inició en Noviembre del 2008
        \newline
        Juan Antonio Infantes Díaz (Dirigido por David Santo Orcero)
        \newline
        Objetivo: Entorno domótico libre.
            \newline
            Tecnologias utilizadas:
                \newline
                Arduino: Hardware libre
                \newline
                Mister House: Software libre
                \newline
                La librería que será implementada en este proyecto, la cual será liberada bajo licencia GPL.
                \newline
            ¿Qué necesitaremos para empezar este proyecto?
                \newline
                Aprender perfectamente como funciona el protocolo X10.
                \newline
                Además de Mister House, ya que será el motor que envíe las instrucciones al Arduino.
                \newline
                Perl, ya que la Mister House está escrito en este lenguaje.
                \newline
                Programación de Arduinos.
                \newline

    + OpenDomo - http://es.opendomo.org/
        \newline
        Inició en 2006
        \newline
        Daniel Lerch y Oriol Palenzuela
        \newline
        Ofrece servicios básicos de todo sistema de control domótico
        \newline
        Surge de la necesidad de unificar las diferentes tecnologías domóticas como: uPnP, X10, EIB, etc, con el protocolo TCP/IP.
        \newline
        Han desarrollado una software domotico bajo Arduino llamado "Domino"
        \newline
        Conexion con Android -> http://es.opendomo.org/android
