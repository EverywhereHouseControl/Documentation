\chapter{Introducción}
En este documento se presentan los posibles riesgos que el proyecto puede llegar a sufrir, así como su nivel de impacto y su gestión. Todos los riesgos descritos pueden verse sometidos a modificaciones por parte del equipo del proyecto.
\section{Propósito}
Este documento pretende servir de guía para resolver los distintos riesgos que pueden llegar a presentarse durante el desarrollo de proyecto. En él, definimos planes de gestión para los riesgos que pueden aparecer con mayor probabilidad o que tienen un mayor impacto en el desarrollo de nuestro proyecto, para así poder minimizar sus consecuencias y normalizar la situación con la máxima calidad posible.
\section{Alcance}
La lista de riesgos expuesta se utilizará durante el desarrollo del proyecto EHC, que tendrá lugar en el curso académico 2013/2014. No obstante, parte de la información que en él se recoge es recomendable que se reutilice para otros proyectos de la asignatura Ingeniería dl Software.
\section{Definición, acrónimos y abreviaciones}
En el documento Glosario se encuentran todas las definiciones, acrónimos y abreviaturas utilizadas a lo largo de esta sección.
\section{Referencias}
\section{Vista general}
El presente documento se estructura en las siguientes subsecciones:
	\begin{itemize}
		\item {\bf Introducción}. En esta sección se muestran los objetivos de este documento y su estructura.
		\item {\bf Resumen de riesgos}. En ella se proporciona una visión general de todos los riesgos que pueden aparecer durante el desarrollo del proyecto.
		\item {\bf Tareas de gestión de riesgos}. Expone la forma de asignar prioridad a los riegos, así como el protocolo a seguir para actualizar el presente documento.
		\item {\bf Organización y responsabilidades}. Determina qué integrantes del grupo se encargan de mantener el presente documento, así como de detectar y proponer planes de gestión para riesgos aún no contemplados en él.                
		\item {\bf Herramientas y técnicas}. Los métodos utilizados para gestionar  adecuadamente los riesgos se expondrán en ésta sección.
		\item {\bf Elementos de riesgo a gestionar}. Describe de manera rigurosa todos los riesgos que pueden aparecer durante el desarrollo de nuestro proyecto.
	\end{itemize}