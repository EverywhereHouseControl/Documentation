\chapter{Tareas de gestión de riesgos}
En la presente sección se describirán todas las tareas de gestión de riesgos, comenzando por la forma en la que se asignarán prioridades a los riesgos, fijando después la forma de actualizar el presente documento.\\

La prioridad de los riegos se asignará en función de:

	\begin{itemize}
		\item La probabilidad de que aparezcan.
		\ item La magnitud del trastorno que provoquen en el proceso de desarrollo.
	\end{itemize}
	
Para actualizar el documento de gestión de riesgos, conviene recordar que la mayoría de los mismos en él recogidos, han ido apareciendo durante el proceso de desarrollo, por tanto, es preciso:

	\begin{itemize}
	\item Pensar en los diferentes tipos de riesgos antes de que puedan aparecer, al principio de cada ciclo.
	\item Desarrollar planes de gestión para los nuevos riesgos e incluirlos en el presente documento tan pronto como sea posible
	\end{itemize}