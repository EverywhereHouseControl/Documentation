\chapter{Elementos de riesgo a gestionar}
\begin{itemize}
\item { \bf Riesgo-R1: Incompatibilidad a la hora de concertar reuniones.}
		\begin{itemize}	
		\item{ \bf Probabilidad}: Frecuente.
		\item{ \bf Nivel de impacto}: Serio.
		\item{ \bf Indicador}: La mayoría de los miembros del equipo no pueden quedar a la misma hora para realizar una reunión.
		\item{ \bf Prevención}: Convocar la reunión con suficiente antelación como para discutir	fechas y horarios.
		\item{ \bf Mitigación}: Adaptar los horarios de todo el equipo para intentar que la reunión sea completa.
		\item{ \bf Plan de contingencia}: El coordinador decidirá una fecha y hora de manera que la mayoría de los miembros pueda asistir.\\
		\end{itemize}
\item { \bf Riesgo-R2: Ausencia del software de desarrollo requerido en los equipos personales.}
		\begin{itemize}	
		\item{ \bf Probabilidad}: Remota.
		\item{ \bf Nivel de impacto}: Serio.
		\item{ \bf Indicador}: Alguno de los miembros de un determinado departamento no dispone del software de desarrollo necesario en su equipo de trabajo.
		\item{ \bf Prevención}: Revisiones periódicas del software instalado,incluyendo versiones, parches y hotfixes.
		\item{ \bf Mitigación}: Asegurar que el software y sus versiones son compatibles con las del resto del equipo en cada dapartamento.
		\item{ \bf Plan de contingencia}:El miembro que se encuentre en esta situación deberá descargar (si fuera necesario) el software correspondiente e instalarlo, así	como las versiones, parches y hotfixes necesarios.\\
		\end{itemize}
\item { \bf Riesgo-R3: Problemas con las tecnologías empleadas debido a su desconocimiento.}
		\begin{itemize}	
		\item{ \bf Probabilidad}: Frecuente.
		\item{ \bf Nivel de impacto}: Serio.
		\item{ \bf Indicador}: Parte de los miembros de uno o varios departamentos desconocen la/s tecnología/s que se usará/n en el proyecto.
		\item{ \bf Prevención}: Realizar tutoriales del mismo y adaptarse a su funcionamiento.
		\item{ \bf Mitigación}: Ninguna.
		\item{ \bf Plan de contingencia}: El coordinador de cada departamento convocará a todos aquellos miembros que desconozcan las tecnologías y no hayan realizado ningún tutorial de las mismas para crear un plan de trabajo orientado a esto mismo.\\
		\end{itemize}
\item { \bf Riesgo-R4: Problemas técnicos de todos o alguno de los miembros del grupo con su equipo o equipos de trabajo.}
		\begin{itemize}	
		\item{ \bf Probabilidad}: Probable.
		\item{ \bf Nivel de impacto}: Serio.
		\item{ \bf Indicador}: Alguno de los miembros no dispone del software necesario para su trabajo, la licencia del mismo ha caducado o el equipo, en general, no funciona como debería.
		\item{ \bf Prevención}: Mantener al día el software, con actualizaciones,parches y hotfixes. Realizar tareas de mantenimiento semanales o mensuales e informar de cualquier problema al coordinador del departamento.
		\item{ \bf Mitigación}: Trabajar con versiones gratuitas del software o usar algún equipo que disponga del software mientras se repara el equipo del miembro.
		\item{ \bf Plan de contingencia}: Si fuese necesario, todo el equipo intervendrá para ayudar a solucionar cualquier problema con el equipo. Si el fallo es debido a las licencias, el coordinador del departamento o en última instancia el director del proyecto hará lo posible por actualizar o adquirir una nueva licencia.\\
		\end{itemize}
\item { \bf Riesgo-R5: Problemas relacionados con el repositorio.}
		\begin{itemize}	
		\item{ \bf Probabilidad}: Ocasional.
		\item{ \bf Nivel de impacto}: Catastrófico/serio.
		\item{ \bf Indicador}: Por causas ajenas al equipo de desarrollo, no se puede acceder al repositorio total o parcialmente.
		\item{ \bf Prevención}: Existencia de una copia de seguridad del código fuente.
		\item{ \bf Mitigación}: Puesta en común del código fuente disponible con la mayor rapidez posible.
		\item{ \bf Plan de contingencia}: Localización y uso de la última copia de seguridad. Se acordará un plan de creación de copias de seguridad, de manera que cada miembro del grupo, además de mantener la suya, mantenga la de otro integrante del grupo.\\
		\end{itemize}
\item { \bf Riesgo-R6: Desacuerdo sobre algún tema decidido anteriormente.}
		\begin{itemize}	
		\item{ \bf Probabilidad}: Improbable.
		\item{ \bf Nivel de impacto}: Serio.
		\item{ \bf Indicador}: Por razones de desarrollo, diseño u otro se llega a la conclusión que alguna toma de decisión importante ha sido errónea.
		\item{ \bf Prevención}: Tomar decisiones con una buena base de manera que se valoren pros y contras.
		\item{ \bf Mitigación}: Tratar de adaptar la decisión.
		\item{ \bf Plan de contingencia}:Adaptar todo lo creado hasta el momento a esta nueva decisión. La adaptación dependerá de la decisión concreta y de los coordinadores.\\
		\end{itemize}
\item { \bf Riesgo-R7: Indecisión frente a alguna de las funcionalidades ya construidas.}
		\begin{itemize}	
		\item{ \bf Probabilidad}: Remota.
		\item{ \bf Nivel de impacto}:
			\begin{itemize}
			\item Crítica si se ha desarrollado la funcionalidad.
			\item Menor si no se ha desarrollado la funcionalidad.
			\end{itemize}
		\item{ \bf Indicador}: Tras haber decidido una funcionalidad del sistema, se considera que no es adecuada y decide cambiarse o eliminarse totalmente.
		\item{ \bf Prevención}: Asegurar una funcionalidad antes de decidir su incorporación al desarrollo.
		\item{ \bf Mitigación}: Tratar de cambiar mínimamente la funcionalidad.
		\item{ \bf Plan de contingencia}: Eliminar total o parcialmente el desarrollo de la funcionalidad y elaborarla de nuevo.\\
		\end{itemize}
\item { \bf Riesgo-R8: Conflictos entre miembros del grupo.}
		\begin{itemize}	
		\item{ \bf Probabilidad}: Remota.
		\item{ \bf Nivel de impacto}: Serio.
		\item{ \bf Indicador}: Dos o más miembros del equipo entran en conflicto por los motivos que fueran.
		\item{ \bf Prevención}: Crear un buen ambiente de trabajo. Crear piña con las reuniones generales así como en las quedadas. En caso de preveer conflicto, arreglarlo como personas adultas o contactar con el coordinador.
		\item{ \bf Mitigación}: Si no se llega a una solución de manera rápida, separar a los miembros implicados, si fuera preciso, del proyecto.
		\item{ \bf Plan de contingencia}: Si tras un tiempo prudencial el conflicto no se ha resuelto, apartar a los miembros implicados del proyecto hasta que se solucione dicho conflicto. Si pasado un tiempo estimado de dos semanas el conflicto sigue vigente, se expulsará a los miembros y se procederá a su sustitución.\\
		\end{itemize}
\item { \bf Riesgo-R9: Pérdida temporal de algún miembro del grupo.}
		\begin{itemize}	
		\item{ \bf Probabilidad}: Remota.
		\item{ \bf Nivel de impacto}: Serio.
		\item{ \bf Indicador}: Uno de los miembros del equipo se ausenta del proyecto por una duración indeterminada y con poco o nulo tiempo de preaviso.
		\item{ \bf Prevención}: Comunicar este tipo de situación al coordinador lo antes posible.
		\item{ \bf Mitigación}: El miembro afectado puede indicar al resto del grupo lo que le resta para completar su tarea.
		\item{ \bf Plan de contingencia}: Asumir el trabajo del miembro por el coordinador correspondiente	o el resto de miembros del equipo.\\
		\end{itemize}
\item { \bf Riesgo-R10: Pérdida definitiva de algún miembro del grupo.}
		\begin{itemize}	
		\item{ \bf Probabilidad}: Remota.
		\item{ \bf Nivel de impacto}: Serio.
		\item{ \bf Indicador}: Uno de los miembros del equipo decide marcharse del proyecto.
		\item{ \bf Prevención}: Intentar que todos los miembros del grupo trabajen a gusto, zanjando las disputas internas tan pronto como aparezcan.
		\item{ \bf Mitigación}: Ninguna.
		\item{ \bf Plan de contingencia}: Dependiendo del número de miembros que se perdiesen se aplicarían diferentes protocolos:
			\begin{itemize}
			\item Pérdida de uno o dos miembros del equipo: Se adaptará la carga de trabajo al resto del grupo, de manera que el grupo pueda seguir realizando el trabajo.
			\item Pérdida de tres o mas miembros del equipo: Se intentará reclutar nuevos miembros de	equipos de mayor tamaño. En caso de no encontrar sustitutos se negociará con el profesor una reducción del tamaño del proyecto o la disolución del equipo.\\
			\end{itemize}
		\end{itemize}
\item { \bf Riesgo-R11: Pérdida del coordinador del grupo.}
		\begin{itemize}	
		\item{ \bf Probabilidad}: Improbable.
		\item{ \bf Nivel de impacto}: Catastrófico.
		\item{ \bf Indicador}: El coordinador general decide abandonar el proyecto por los motivos que sean.
		\item{ \bf Prevención}: Mantener el buen hacer del proyecto en general, avanzar y trabajar según lo establecido.
		\item{ \bf Mitigación}: Si el coordinador decide abandonar el proyecto, deberá buscar un sustituto temporal. O en su defecto cumplir con las tareas asignadas hasta la elección del nuevo coordinador.
		\item{ \bf Plan de contingencia}: Se comenzará buscando voluntarios para el puesto de coordinador. Si una persona quiere y es aceptada por el resto de integrantes quedará como coordinador. Si hay más de un candidato, se someterá a votación. Si no hay ningún candidato, se escogerá al integrante con mayores dotes de liderazgo del grupo.\\
		\end{itemize}
\item { \bf Riesgo-R12: Cambio de los requisitos en el proyecto.}
		\begin{itemize}	
		\item{ \bf Probabilidad}: Improbable.
		\item{ \bf Nivel de impacto}: Catastrófico.
		\item{ \bf Indicador}: Por motivos técnicos o de diseño, se decide cambiar gran parte de los requisitos del proyecto.
		\item{ \bf Prevención}: Realizar una especificación de requisitos fuerte y bien planificada. 
		\item{ \bf Mitigación}: Tratar de adaptar los nuevos requisitos al trabajo ya desarrollado.
		\item{ \bf Plan de contingencia}: Cambiar todo el desarrollo llevado hasta el momento para acometer los nuevos requisitos.\\
		\end{itemize}
\item { \bf Riesgo-R13: Necesidad de introducir cambios en la arquitectura del proyecto.}
		\begin{itemize}	
		\item{ \bf Probabilidad}: Improbable.
		\item{ \bf Nivel de impacto}: Catastrófico.
		\item{ \bf Indicador}: Por alguna razón técnica, se hace indispensable cambiar la arquitectura del proyecto (clases, interfaces, relaciones, patrones, etc).
		\item{ \bf Prevención}: Realización de un buen y claro desarrollo de la arquitectura software del proyecto, teniendo en cuenta perjuicios y bondades de las diferentes decisiones.
		\item{ \bf Mitigación}: Detener el proceso de desarrollo y preparar una lista de todo el código y de toda la documentación disponibles.
		\item{ \bf Plan de contingencia}: Reescritura total o parcial del código implicado en estos cambios.\\
		\end{itemize}
\item { \bf Riesgo-R14: Conexión remota sobresaturada.}
		\begin{itemize}	
		\item{ \bf Probabilidad}: Frecuente.
		\item{ \bf Nivel de impacto}: Serio.
		\item{ \bf Indicador}: Se pierde la conexión remota con el sistema o es demasiado lenta debido a la saturación de la red.
		\item{ \bf Prevención}: Limitar el número de usuarios que pueden acceder al servidor para garantizar su rendimiento.
		\item{ \bf Mitigación}: Ninguno.
		\item{ \bf Plan de contingencia}: Para evitar que el rendimiento del servidor se vea afectada, se interrumpiría hasta que sean tratadas todas las peticiones.\\
		\end{itemize}
\item { \bf Riesgo-R15: Problemas al ensamblar distintas partes del proyecto.}
		\begin{itemize}	
		\item{ \bf Probabilidad}: Frecuente.
		\item{ \bf Nivel de impacto}: Menor.
		\item{ \bf Indicador}: Al ensamblar varias partes finalizadas, diversas funcionalidades, el sistema no reacciona como es de esperar (desde el punto de vista técnico).
		\item{ \bf Prevención}: Intentar, tanto como sea posible, que cada parte sea técnicamente compatible a través de pruebas y tests.
		\item{ \bf Mitigación}: Ninguna.
		\item{ \bf Plan de contingencia}: Revisar el código de cada una de las partes, asegurando que sea compatible. Reescribir, si fuese necesario, bloques de código para solucionar el problema. Aislar diversas partes para realizar micro-pruebas.\\
		\end{itemize}
\item { \bf Riesgo-R16: Cambio de la arquitectura del sistema.}
		\begin{itemize}	
		\item{ \bf Probabilidad}: Improbable.
		\item{ \bf Nivel de impacto}: Catastrófico.
		\item{ \bf Indicador}: El sistema no funciona como debería y se debe a una cuestión de diseño.
		\item{ \bf Prevención}: Diseñar la arquitectura previamente intentando abarcar los problemas que se puedan tener en el futuro.
		\item{ \bf Mitigación}: Dedicar un esfuerzo grande por parte de un subgrupo de integrantes del equipo a desarrollar la arquitectura, siendo esta su única función.
		\item{ \bf Plan de contingencia}: Destinar cuantos recursos humanos sean necesarios para rediseñar e implementar la arquitectura con la máxima prioridad posible.\\
		\end{itemize}
\item { \bf Riesgo-R17: Interfaces demasiado complejas.}
		\begin{itemize}	
		\item{ \bf Probabilidad}: Remota.
		\item{ \bf Nivel de impacto}: Serio.
		\item{ \bf Indicador}: El uso de la aplicación no resulta intuitivo. 
		\item{ \bf Prevención}: Implementar interfaces amigables, sencillas e intuitivas consiguiendo que en ellas se presenten de forma clara las diversas opciones y funcionalidades que ofrece el sistema.
		\item{ \bf Mitigación}: Desarrollar maquetas y presentárselas a personas ajenas al proyecto para que expresen su opinión sobre las mismas.
		\item{ \bf Plan de contingencia}: Rediseñar las interfaces teniendo como objetivo final la claridad de las mismas.\\ 
		\end{itemize}
\item { \bf Riesgo-R18: Problemas de compatibilidades con diferentes sistemas operativos (incluidas sus diferentes versiones).}
		\begin{itemize}	
		\item{ \bf Probabilidad}: Ocasional.
		\item{ \bf Nivel de impacto}: Crítico.
		\item{ \bf Indicador}: Las pruebas no se desarrollan como deberían en diferentes SO.
		\item{ \bf Prevención}: Asegurarse de que el código es válido para cada versión de cada SO (biblioteca del SO, arquitecturas, etc) creando, si fuera necesario, diferentes versiones del desarrollo.
		\item{ \bf Mitigación}: Dejar cerrado los SO y versiones de los mismos en los que el proyecto podrá ejecutarse (en la arquitectura software).
		\item{ \bf Plan de contingencia}: Crear diferentes versiones del software desarrollado para que funcione en los SO en los que no lo haga.\\
		\end{itemize}
\item { \bf Riesgo-R19: Cambio de la plataforma de desarrollo por razones técnicas.}
		\begin{itemize}	
		\item{ \bf Probabilidad}: Remota.
		\item{ \bf Nivel de impacto}: Catastrófico.
		\item{ \bf Indicador}: Llegar a un punto del desarrollo en el que, técnicamente, el software usado hasta ese momento no puede abarcar las funcionalidades del sistema.
		\item{ \bf Prevención}: Especificar las diferentes funcionalidades en base a las posibilidades del software de desarrollo.
		\item{ \bf Mitigación}: Realizar pequeñas pruebas de las funcionalidades antes de validarlas por completo.
		\item{ \bf Plan de contingencia}: Dependiento del tipo que sea la funcionalidad se apliacarán diferentes protocolos:
			\begin{itemize}
			\item Si la funcionalidad no es esencial y no se ha desarrollado parte del proyecto en base a ésta, cambiar la funcionaliadad para que se ajuste a las posibilidades técnicas.
			\item Si la funcionalidad es esencial o se han desarrollado diversas partes del proyecto en base a ésta, se migrará todo el código necesario.\\
			\end{itemize}
		\end{itemize}
\item { \bf Riesgo-R20: Cambio radical de la visión del proyecto por ser este de dimensiones inalcanzables.}
		\begin{itemize}	
		\item{ \bf Probabilidad}: Improbable.
		\item{ \bf Nivel de impacto}: Catastrófico.
		\item{ \bf Indicador}: En un punto avanzado del proyecto, se detecta que este es demasiado grande como para abarcarlo con los medios disponibles.
		\item{ \bf Prevención}: Dedicar suficiente tiempo para desarrollar una coherente visión del proyecto, procurando que todos los miembros del equipo intervengan en su creación.
		\item{ \bf Mitigación}: Eliminar una o varias características para disminuir el volumen del proyecto.
		\item{ \bf Plan de contingencia}: Se introducirán los cambios oportunos para que el proyecto pueda llevarse a cabo.\\
		\end{itemize}
\item { \bf Riesgo-R21: Solicitud de presentaciones de cualquier tipo.}
		\begin{itemize}	
		\item{ \bf Probabilidad}: Remota.
		\item{ \bf Nivel de impacto}: Menor.
		\item{ \bf Indicador}: Algún medio externo solicita una presentación del proyecto.
		\item{ \bf Prevención}: Se llevará la documentación lo más al día posible para que la recopilación de información sea rápida.
		\item{ \bf Mitigación}: Frecuente actualización de la documentación.
		\item{ \bf Plan de contingencia}: Un miembro del equipo será el encargado de recuperar toda la información para presentarla al medio.\\
		\end{itemize}
\item { \bf Riesgo-R22: Proximidad de exámenes o trabajos académicos de cualquier miembro.}
		\begin{itemize}	
		\item{ \bf Probabilidad}: Frecuente.
		\item{ \bf Nivel de impacto}: Serio.
		\item{ \bf Indicador}: Proximidad de fechas de exámenes y/o entregas de trabajos indicados por los miembros del equipo.
		\item{ \bf Prevención}: Saber de antemano, y con precisión, qué fechas serán estas.
		\item{ \bf Mitigación}: Suavizar la cantidad de trabajo durante esas fechas.
		\item{ \bf Plan de contingencia}: En caso extremo, paralizar el proyecto hasta que la mayoría de los miembros vuelvan a estar activos.\\
		\end{itemize}
\item { \bf Riesgo-R23: Adelanto de la fecha de entrega del proyecto o entregas y presentaciones parciales.}
		\begin{itemize}	
		\item{ \bf Probabilidad}: Remota.
		\item{ \bf Nivel de impacto}: Serio.
		\item{ \bf Indicador}: Por diversos motivos, se decide adelantar las fechas de presentaciones, finalización de diversos módulos, o del proyecto en sí.
		\item{ \bf Prevención}: Adelantar tanto trabajo como se pueda, cumpliendo con las planificaciones.
		\item{ \bf Mitigación}: Negociar la fecha de entrega.
		\item{ \bf Plan de contingencia}: Se revisará lo que falta por terminar del proyecto y se fijarán prioridades. Si no hubiese tiempo para terminarlo todo, se eliminarían las funciones menos prioritarias.\\
		\end{itemize}
\item { \bf Riesgo-R24: Imposibilidad de entregar toda la funcionalidad planificada por falta de tiempo.}
		\begin{itemize}	
		\item{ \bf Probabilidad}: Frecuente.
		\item{ \bf Nivel de impacto}: Menor.
		\item{ \bf Indicador}: Estando cerca de la fecha de entrega, no se ve plausible terminar toda funcionalidad planificada.
		\item{ \bf Prevención}: Realizar una planificación realista y trabajar con vistas a la misma.
		\item{ \bf Mitigación}: Hacer un gran esfuerzo por parte del equipo para cumplir con lo planificado.
		\item{ \bf Plan de contingencia}: Los módulos se simplificarán en la medida de lo posible para conseguir entregar a tiempo. Se descartarán partes de la funcionalidad del sistema si es preciso.\\
		\end{itemize}
\item { \bf Riesgo-R25: Abandono total del proyecto.}
		\begin{itemize}	
		\item{ \bf Probabilidad}: Improbable.
		\item{ \bf Nivel de impacto}: Catastrófico.
		\item{ \bf Indicador}: El grupo decide disolverse y el proyecto no se finaliza.
		\item{ \bf Prevención}: Motivar al grupo cada semana para que el trabajo no sea tan pesado.
		\item{ \bf Mitigación}: Tener una buena actitud hacia el proyecto (conseguida la mayoría de las veces con el propio avance del proyecto). 
		\item{ \bf Plan de contingencia}: Buscar otro equipo para continuar con el proyecto o en el peor de los casos empezar uno nuevo.\\
		\end{itemize}
\end{itemize}