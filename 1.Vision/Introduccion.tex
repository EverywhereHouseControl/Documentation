\chapter{Visión}

\section{Propósito}
    El presente documento tiene como objetivo mostrar al lector el objetivo, requisitos y funcionalidades que ofrece el sistema \textit{Everywhere House Control} (EHC). \par
    Se recoge, analiza y define las necesidades de alto nivel y las características de un sistema que gestiona y automatiza los recursos del hogar (o centro de trabajo) a través de un dispositivo móvil de forma remota.

\section{Alcance}
    El sistema EHC será una aplicación ejecutable en cualquier dispositivo móvil que permitirá administrar y controlar los recursos del hogar (o centro de trabajo), ya sea tanto dentro como fuera del mismo.

\section{Definiciones, acrónimos y abreviaciones}
    \subsection{Del sistema}
        \begin{itemize}
            \item { \bf EHC}: Siglas correspondientes al nombre del sistema desarrollado (Everywhere House Control)
            \item { \bf Scrum}: Marco de trabajo para la gestión y desarrollo del software. Se trata de un proceso que se aplican de manera regular un conjunto de buenas prácticas para trabajar colaborativamente (en equipo). Se realizan entregas parciales y regulares del producto final, por lo que este tipo de proceso está especialmente recomendado para entornos complejos, donde se necesitan tener resultados pronto. \href{http://es.wikipedia.org/wiki/Scrum}{Scrum} se ejecuta en bloques temporales cortos y fijos (puede ser semanales, quincenales...), donde cada uno de ellos debe de producir un resultado que beneficiará al incremento del producto final. 
            \item { \bf Entorno EHC}: Hogar o lugar de trabajo donde se implanta el sistema.
			\item { \bf }	Prototipo: un producto preliminar, parcial o total, que servirá para probar la funcionalidad y sobre el que se desarrollará los test de validación y evaluación. 
			\item { \bf }	Mejora o evolución de un prototipo: cambios en la funcionalidad, arquitectónica o especificaciones, que buscan desarrollar un producto final de calidad.
			\item { \bf	} Pm: personas mes.
			\item { \bf	} Pd: personas día.			
			\item { \bf	} RUP: (Rational Unified Process) proceso unificado de desarrollo de software.
			\item { \bf	} Diagrama burn down: gráfica del trabajo por hacer en un proyecto en el tiempo.
			\item { \bf	} Sprint: periodo temporal con un conjunto de tareas asociadas a realizarse en el mismo.
			\item { \bf	} Herramienta CASE: programa o aplicación que nos ayuda en la administración del proyecto.
			\item { \bf	} Stakeholders: aquellas personas a quienes pueden afectar o son afectadas por las actividades de una empresa.            
        \end{itemize}
    \subsection{De tecnología}
    	\begin{itemize}
    		\item {\bf PnP }: es un conjunto de protocolos de comunicación que permite a periféricos en red, como ordenadores personales, impresoras, pasarelas de Internet, puntos de acceso Wi-Fi y dispositivos móviles, descubrir de manera transparente la presencia de otros dispositivos en la red y establecer servicios de red de comunicación, compartición de datos y entretenimiento.
    		\item {\bf X10 }: Es un protocolo de comunicaciones para el control remoto de dispositivos eléctricos, que utiliza la línea eléctrica (220V o 110V) preexistente, para transmitir señales de control entre equipos de automatización del hogar (domótica) en formato digital.
    		\item {\bf EIB}: (European Installation Bus) Es un sistema domótico que se desarrollo bajo el
    		amparo de la Unión Europea, con el objetivo de disminuir el número de importaciones de
    		productos del mismo tipo provenientes de los mercados japoneses y norteamericanos,
    		donde este tipo de tecnologías estaban más desarrolladas.
    		\item {\bf TCP/IP}: La familia de protocolos de Internet es un conjunto de protocolos de red en los que se basa Internet y que permiten la transmisión de datos entre computadoras. En ocasiones se le denomina conjunto de protocolos TCP/IP, en referencia a los dos protocolos más importantes que la componen: Protocolo de Control de Transmisión (TCP) y Protocolo de Internet (IP), que fueron dos de los primeros en definirse, y que son los más utilizados de la familia. 
    	\end{itemize}
    	

\section{Referencias}
    \begin{itemize}
        \item Glosario
        \item Plan de desarrollo software
        \item Scrum
        \item Diagrama de casos de uso
        \item Guías de estilo
    \end{itemize}
