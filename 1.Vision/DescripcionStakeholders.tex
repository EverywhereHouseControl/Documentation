\chapter{Descripción de Stakeholders y usuarios}
    Para proveer de una forma efectiva el servicio y que se ajuste a las necesidades de los usuarios, es necesario identificar e involucrar a todos los participantes en el proyecto como parte del proceso de modelado de requerimientos. También es necesario identificar a los usuarios del sistema y asegurarse de que el conjunto de participantes en el proyecto los representa adecuadamente. Esta sección muestra un perfil de los participantes y de los usuarios involucrados en el proyecto, así como los problemas más importantes que éstos perciben para enfocar la solución propuesta hacia ellos. No describe sus requisitos específicos pero proporciona la justificación de por qué estos requisitos son necesarios.

\section{Resumen de \textit{Stakeholders}}
    \begin{tabular}{|p{4.5cm}|p{4cm}|p{8cm}|}
        \hline \textbf{Nombre} &  \textbf{Descripción} & \textbf{Responsabilidades} \\
        \hline Morales Lozano, Álvaro & Representante global de la empresa EHC. &
		Seguimiento del desarrollo del proyecto a nivel global y la aprobación de requisitos y funcionalidades a nivel global. \\
        \hline Vicente Sánchez, Victor & Representante del departamento de Software. &
        Seguimiento del desarrollo del proyecto a nivel software y la aprobación de requisitos y funcionalidades.\\
        \hline Gutiérrez, Héctor & Representante del departamento de Hardware. &
		Seguimiento del desarrollo del proyecto a nivel hardware y la aprobación de requisitos y funcionalidades. \\
        \hline Ladrón De Guevara, Alejandro & Representante del departamento de Servidor.&
        Seguimiento del desarrollo del proyecto a nivel servidor y la aprobación de requisitos y funcionalidades. \\
        \hline Maldonado Lenis, Miguel Alexander & Desarrollador del departamento de Software. &
		Desarrollador de la aplicación de Android y responsable de la conectividad entre la aplicación y el servidor. \\        
        \hline Tirado Caamaño, Colin & Representante del departamento de documentación. &
		Mantener actualizada la documentación de los distintos departamentos y la organización y maquetación de la documentación. \\  
        \hline Guzman, Fernando & Desarrollador del departamento de Hardware & 
        Implementación y desarrollo de los distintos prototipos Hardware. \\                    
        \hline Ochoa, Victor & Desarrollador del departamento de Hardware &
		Implementación y desarrollo de los distintos prototipos Hardware. \\                            
        \hline Rey, José Antonio & Desarrollador del departamento del servidor &
		Responsable del manteniento y gestión de los distintos sitios contratados por la empresa y la instalación de la infraestructura en el servidor. \\                            
        \hline Saavedra, Luis Antonio & Desarrollador del departamento del servidor &
		Responsable de la implementación de los servicios del servidor. \\
        \hline Peinado Gil, Federico & Departamento de control de calidad &
TBD
		\\		
		\hline Hassan Collado, Samer & Departamento de control de calidad &
		TBD		
		\\		
		\hline                            
    \end{tabular}

\section{Resumen de usuarios}
    \begin{tabular}{|p{3cm}|p{7cm}|p{6cm}|}
        \hline \textbf{Nombre} &  \textbf{Descripción} & \textbf{StakeHolder} \\
        \hline Administrador & Encargado de la instalación del sistema EHC en el cliente. Se encargará de  instalar/configurar el sistema en el entorno EHC. & Gutiérrez, Hector \\
        \hline Usuario & Participante del entorno EHC &   Ladrón De Guevara, Alejandro \par Vicente Sánchez, Victor \\
        \hline
    \end{tabular}

\section{Entorno de usuarios}
    Los usuarios entrarán al sistema identificándose sobre cualquier dispositivo móvil que tenga instalada la aplicación de gestión del sistema EHC o a través de la aplicación web. Tras este paso podrán controlar multitud de tareas y acciones que ofrecen los distintos recursos del entorno EHC navegando a través de interfaces ágiles y amigables.

\section{Perfil de \textit{Stakeholders}}
    \subsection{Departamento de Software}
        \begin{tabular}{|p{4cm}|p{12cm}|}
            \hline \textbf{Representante} & Vicente Sánchez, Victor. \\
            \hline \textbf{Descripción} & Representante del Departamento de Software. \\
            \hline \textbf{Tipo} &  [Experto] Estudiante de sistemas software. \\
            \hline \textbf{Responsabilidades} &  Llevar a cabo un seguimiento del desarrollo del proyecto y aprobación de  los requisitos y funcionalidades del sistema a nivel software.\\
            \hline \textbf{Criterio de éxito} &  [A definir por el cliente]\\
            \hline \textbf{Grado de participación} &  Revisión de requerimientos, desarrollo y estructura del sistema a nivel software.\\
            \hline \textbf{Comentarios} &  Ninguno\\
            \hline
        \end{tabular}

        \begin{tabular}{|p{4cm}|p{12cm}|}
            \hline \textbf{Representante} &  Maldonado Lenis, Miguel Alexander. \\
            \hline \textbf{Descripción} &  Componente del Departamento de Software.\\
            \hline \textbf{Tipo} &  [Experto] Estudiante de sistemas software.\\
            \hline \textbf{Responsabilidades} &  Desarrollar requisitos y funcionalidades del sistema a nivel software.\\
            \hline \textbf{Criterio de éxito} & [A definir por el cliente]\\
            \hline \textbf{Grado de participación} & Funcional. Llevando a cabo en los plazos planificados las tareas que le van siendo asignadas por el coordinador.\\
            \hline \textbf{Comentarios} &  Ninguno.\\
            \hline
        \end{tabular}
        
		\begin{tabular}{|p{4cm}|p{12cm}|}
            \hline \textbf{Representante} & Tirado, Colin \\
            \hline \textbf{Descripción} & Componente del Departamento de Software. \\
            \hline \textbf{Tipo} & [Experto] Estudiante de sistemasde gestión. \\
            \hline \textbf{Responsabilidades} & Llevar a cabo un seguimiento del desarrollo del proyecto con la finalidad de recopilar información para elaborar y mantener la documentación del mismo. Y desarrollar requisitos y funcionalidades del sistema a nivel software. \\
            \hline \textbf{Criterio de éxito} & [A definir por el cliente] \\
            \hline \textbf{Grado de participación} & Revisión, elaboración y mantenimiento de la documentación. Y realización en los plazos planificados de las tareas que le van siendo asignadas por el coordinador.  \\
            \hline \textbf{Comentarios} &  Ninguno. \\
            \hline
        \end{tabular}

    \subsection{Departamento de Hardware}

        \begin{tabular}{|p{4cm}|p{12cm}|}
            \hline \textbf{Representante} & Morales Lozano, Álvaro \\
            \hline \textbf{Descripción} &  Componente del Departamento de Hardware. Y representante global de la empresa EHC. \\
            \hline \textbf{Tipo} &  [Experto] Estudiante de gestión de sistemas.\\
            \hline \textbf{Responsabilidades} &  Encargado de mostrar las necesidades de cada usuario del sistema. Además, llevar a cabo un seguimiento del desarrollo del proyecto y aprobación de  los requisitos y funcionalidades del sistema a nivel global y a nivel hardware.\\
            \hline \textbf{Criterio de éxito} &  [A definir por el cliente]\\
            \hline \textbf{Grado de participación} &  Revisión de requerimientos, desarrollo y estructura del sistema a nivel global. Y realización en los plazos planificados de las tareas que le van siendo asignadas por el coordinador.\\
            \hline \textbf{Comentarios} &  Ninguno.\\
            \hline
        \end{tabular}

        \begin{tabular}{|p{4cm}|p{12cm}|}
            \hline \textbf{Representante} &  Gutiérrez, Héctor\\
            \hline \textbf{Descripción} &  Representante del Departamento de Hardware. \\
            \hline \textbf{Tipo} &  [Experto] Estudiante de sistemas hardware.\\
            \hline \textbf{Responsabilidades} &  Llevar a cabo un seguimiento del desarrollo del proyecto y aprobación de  los requisitos y funcionalidades del sistema a nivel hardware. \\
            \hline \textbf{Criterio de éxito} &  [A definir por el cliente]\\
            \hline \textbf{Grado de participación} &  Revisión de requerimientos, desarrollo y estructura del sistema a nivel hardware.\\
            \hline \textbf{Comentarios} &  Ninguno\\
            \hline
        \end{tabular}
        
        \begin{tabular}{|p{4cm}|p{12cm}|}
            \hline \textbf{Representante} & Guzman, Fernando \\
            \hline \textbf{Descripción} & Componente del Departamento de Hardware. \\
            \hline \textbf{Tipo} & [Experto] Estudiante de sistemas hardware. \\
            \hline \textbf{Responsabilidades} & Desarrollar requisitos y funcionalidades del sistema a nivel hardware. \\
            \hline \textbf{Criterio de éxito} & [A definir por el cliente] \\
            \hline \textbf{Grado de participación} & Funcional. Llevando a cabo en los plazos planificados las tareas que le van siendo asignadas por el coordinador. \\
            \hline \textbf{Comentarios} &  Ninguno \\
            \hline
        \end{tabular}
        
		\begin{tabular}{|p{4cm}|p{12cm}|}
            \hline \textbf{Representante} & Ochoa, Victor \\
            \hline \textbf{Descripción} & Componente del Departamento de Hardware. \\
            \hline \textbf{Tipo} & [Experto] Estudiante de sistemas de gestión. \\
            \hline \textbf{Responsabilidades} & Llevar a cabo un seguimiento del desarrollo del proyecto con la finalidad de recopilar información para elaborar y mantener la documentación del mismo. Y desarrollar requisitos y funcionalidades del sistema a nivel hardware. \\
            \hline \textbf{Criterio de éxito} & [A definir por el cliente] \\
            \hline \textbf{Grado de participación} & Revisión, elaboración y mantenimiento de la documentación. Y realización en los plazos planificados de las tareas que le van siendo asignadas por el coordinador. \\
            \hline \textbf{Comentarios} &  Ninguno. \\
            \hline
        \end{tabular}

    \subsection{Departamento de Servidor}
        \begin{tabular}{|p{4cm}|p{12cm}|}
            \hline \textbf{Representante} &  Ladrón De Guevara, Alejandro\\
            \hline \textbf{Descripción} & Representante del Departamento de Servidor. \\
            \hline \textbf{Tipo} &  [Experto] Estudiante de sistemas con servidores. \\
            \hline \textbf{Responsabilidades} & Llevar a cabo un seguimiento del desarrollo del proyecto y aprobación de  los requisitos y funcionalidades del sistema a nivel del servidor. \\
            \hline \textbf{Criterio de éxito} & [A definir por el cliente] \\
            \hline \textbf{Grado de participación} & Revisión de requerimientos, desarrollo y estructura del sistema a nivel del servidor. \\
            \hline \textbf{Comentarios} &  Ninguno. \\
            \hline
        \end{tabular}

        \begin{tabular}{|p{4cm}|p{12cm}|}
            \hline \textbf{Representante} & Rey, José Antonio \\
            \hline \textbf{Descripción} & Componente del Departamento de Servidor  \\
            \hline \textbf{Tipo} & [Experto] Estudiante de sistemas con servidores. \\
            \hline \textbf{Responsabilidades} & Desarrollar requisitos y funcionalidades del sistema a nivel del servidor. \\
            \hline \textbf{Criterio de éxito} & [A definir por el cliente] \\
            \hline \textbf{Grado de participación} & Funcional. Llevando a cabo en los plazos planificados las tareas que le van siendo asignadas por el coordinador. \\
            \hline \textbf{Comentarios} & Ninguno. \\
            \hline
        \end{tabular}

        \begin{tabular}{|p{4cm}|p{12cm}|}
            \hline \textbf{Representante} & Saavedra, Luis Antonio \\
            \hline \textbf{Descripción} & Componente del Departamento de Servidor \\
            \hline \textbf{Tipo} & [Experto] Estudiante de sistemas con servidores. \\
            \hline \textbf{Responsabilidades} & Desarrollar requisitos y funcionalidades del sistema a nivel del servidor. \\
            \hline \textbf{Criterio de éxito} & [A definir por el cliente] \\
            \hline \textbf{Grado de participación} & Funcional. Llevando a cabo en los plazos planificados las tareas que le van siendo asignadas por el coordinador. \\
            \hline \textbf{Comentarios} & Ninguno. \\
            \hline
        \end{tabular}

\section{Tipos de usuario}

    \subsection{Administrador}
        \begin{tabular}{|p{4cm}|p{12cm}|}
            \hline \textbf{Representante} & Administrador \\
            \hline \textbf{Descripción} & Participante directo de la empresa EHC que se encarga de la instalación y el aprendizaje inicial del sistema EHC en el entorno instalado. \\
            \hline \textbf{Tipo} & Usuario casual del sistema. \\
            \hline \textbf{Responsabilidades} & Responsable de la instalación y del funcionamiento del sistema. \\
            \hline \textbf{Criterio de éxito} & [A definir por el cliente] \\
            \hline \textbf{Grado de participación} & Instalación del sistema en el lugar contratado. \par
            Mostrar y aconsejar al usuario como usar el sistema. \\
            \hline \textbf{Comentarios} & Ninguno. \\
            \hline
        \end{tabular}

    \subsection{Cliente}
        \begin{tabular}{|p{4cm}|p{12cm}|}
            \hline \textbf{Representante} & Cliente. \\
            \hline \textbf{Descripción} & Usuario (o empresa) que desee  tener de manera centralizada el control de los recursos del hogar (o centro de trabajo). \\
            \hline \textbf{Tipo} & Usuario casual del sistema. \\
            \hline \textbf{Responsabilidades} & Uso correcto de la instalación del sistema EHC \\
            \hline \textbf{Criterio de éxito} & [A definir por el cliente] \\
            \hline \textbf{Grado de participación} & [A definir por el cliente] \\
            \hline \textbf{Comentarios} & Ninguno. \\
            \hline
        \end{tabular}

\section{Alternativas y competencias}
    A continuación, se mostrará una lista de las principales alternativas al sistema:\\
    \begin{itemize}
    \item {\bf OpenDomotica} -  https://opendomotica.wordpress.com/ \par
	Se trata de un proyecto que tiene como objetivo crear un entorno domótico con software libre que compita con el actual mercado de la domótica. Como aporte interesante de este proyecto, es la tecnología que se usa para la comunicación entre los distintos dispositivos: X10, que se trata de un protocolo estándar para la transmisión por la linea eléctrica. Este aporte le da un gran valor al proyecto, ya que con esta tecnologia, si tenemos una toma de red eléctrica, tenemos una toma de red domótica.
            \begin{tabbing}
            ---- \= ------ \= ----- \= ----- \...\kill
            \>\> {\bf Autor/es}: Juan Antonio Infantes Díaz (Dirigido por David Santo Orcero)\\
            \>\> {\bf Inicio}: Noviembre del 2008.\\
            \>\> {\bf Objetivo}: Entorno domótico libre.\\
            \>\> {\bf Tecnologías utilizadas}:\\
            \>\>\> {\bf Arduino}: Hardware libre.\\
            \>\>\> {\bf Mister House}: Software libre.\\
            \end{tabbing}
               
     \item {\bf OpenDomo} - http://es.opendomo.org/ \par
Es un sistema de control domótico libre y seguro. Se centra en ofrecer una fácil accesibilidad y en la seguridad que puede dar al cliente al usar este servicio (a través de cámaras de video, sensores de movimiento...). 
     		\begin{tabbing}
     		---- \= ------ \= ----- \= ----- \...\kill
            \>\> {\bf Autor/es}: Daniel Lerch y Oriol Palenzuela.\\
            \>\> {\bf Inicio}: 2006.\\
            \>\> {\bf Objetivo}: Ofrece servicios básicos de todo sistema de control domótico.\\
            \>\> {\bf Origen}: Surgió de la necesidad de unificar las diferentes tecnologías domóticas\\\>\>\> como: uPnP, X10, EIB, etc, con el protocolo TCP/IP.\\
            \>\> {\bf Comentarios}: Han desarrollado una software domotico bajo Arduino llamado\\\>\>\> "Domino".\\
            \end{tabbing}
    \end{itemize}
    