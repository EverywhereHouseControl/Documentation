\chapter{Descripción global del producto}

\section{Perspectiva del producto}
    El producto a desarrollar es un sistema que gestiona y automatiza los recursos de un entorno EHC a través de un dispositivo móvil de forma remota.

\section{Resumen de características}
    A continuación se mostrará un listado de beneficios que obtendrá el cliente a partir del producto: \\ \par
    \begin{tabular}{|p{8cm}|p{8cm}|}
        \hline \textbf{Beneficio del cliente} & \textbf{Características que lo apoyan} \\
        \hline  Confort. & Incremento de las posibilidades de control de los
        propios equipos e instalaciones domésticas. De esta manera este control se traduce en un incremento de la calidad de vida.\\
         \hline Seguridad. &  El sistema EHC puede contar con detectores de presencia, cámaras, sensores de gas, detectores de humo, etc. Si el sistema encuentra alguna anomalidad en el entorno avisará al usuario y efectuará acciones preventivas y/o correctivas.\\
         \hline Ahorro energético. & El sistema EHC puede desenchufar aquellos aparatos que no son necesarios cuando no estamos en casa. Además se puede programar el sistema para que se encienda la climatización momentos antes de llegar al hogar reduciendo el consumo al estrictamente necesario.\\
        \hline Accesibilidad. & Se incluyen las aplicaciones o instalaciones de control remoto del entorno que favorecen la autonomía personal de personas con limitaciones funcionales, o discapacidad.\\
        \hline
    \end{tabular}

\section{Suposiciones y dependencias}
    A continuación se muestra una lista de suposiciones y dependencias del sistema que afectan al sistema y al cumplimiento de de lo especificado en los documentos del proyecto: \\ \par
    \begin{tabular}{|p{4cm}|p{6cm}|p{6cm}|}
        \hline \textbf{Categoría} &  \textbf{Comentario} & \textbf{Limitación si no se cumple} \\
        \hline Suposición. & El usuario tiene en el hogar conexión a Internet & El usuario solo podrá usar el sistema  EHC de forma local. \\
        \hline Dependencia. & El hogar dispone de forma ininterrumpida corriente eléctrica. & El sistema EHC no estará disponible en esos periodos sin corriente eléctrica. \\
        \hline
    \end{tabular}

% %\section{Costo y precio}
% %    TBD

\section{Licencia e instalación}
    La instalación y configuración inicial será realizada por el personal de la empresa. Dicha instalación y configuración no solo se centra en el montaje de los elementos hardware necesarios para el funcionamiento del sistema, sino también en la configuración de la red de comunicación entre los distintos dispositivos que se encuentran en el entorno EHC. \par

    Tras dicha instalación, se entregará al usuario una licencia de uso del sistema EHC con sus correspondientes claves de identificación y uso del sistema.
