\chapter{Posicionamiento}

\section{Perspectivas del producto}
    El sistema EHC será un producto dise\~nado para la administración y el control de los recursos del hogar (o centro de trabajo), ya sea tanto dentro como fuera del mismo. Además permitirá que dichas tareas (administración y control) se hagan de una manera centralizada a trav\'es de un único dispositivo.

    EHC se servirá de un servidor y una base de datos para la gestión y administración de perfiles de usuario así como la conexión de los mismos al sistema. La conexión de los usuarios al sistema EHC podrá establecerse tanto dentro como fuera del lugar en el que est\'e implantado el sistema.

    El producto tendrá una aplicación ejecutable en cualquier dispositivo móvil, así como su correspondiente versión web y PC. Dicha aplicación se compondrá de interfaces gráficas intuitivas, sencillas y amigables para cualquier tipo de usuario.

\section{Oportunidad de negocio}
    Este sistema permitirá al usuario tener un control absoluto en el entorno EHC sobre los recursos del mismo (tales como; activar calefacción, subir/bajar persianas, grabar programas de televisión, apagar/encender luces, apagar/encender enchufes, etc.), pudiendo acceder de manera fácil y rápida a cualquier recurso gracias a interfaces gráficas sencillas y amigables.

\section{Sentencia que define el problema}
    \begin{tabular}{|p{6cm}|p{10cm}|}
        \hline \textbf{El problema de} & Controlar y gestionar una serie de recursos un lugar determinado.\\
        \hline \textbf{Afecta a} & Toda persona/empresa que necesite controlar o gestionar un lugar determinado de forma remota.\\
        \hline \textbf{El impacto asociado es} & Controlar desde uno o varios dispositivos aquellos elementos del entorno EHC \\
        \hline \textbf{Una solución adecuada sería} & Automatizar ciertos elementos críticos para el usuario/empresa del entorno EHC para que después sean controlados a través de una aplicación móvil o web desde cualquier lugar. \par
        Este control sobre el entorno EHC puede ser de forma directa a través de Internet o a través de ordenes planificadas gestionadas anteriormente por el usuario.\\
        \hline
    \end{tabular}

\section{Sentencia que define la posición del producto}
    \begin{tabular}{|p{6cm}|p{10cm}|}
        \hline \textbf{Para} &  Usuario (o empresa) que desee  tener de manera centralizada el control de los recursos del entorno EHC.\\
        \hline \textbf{Quienes} & Necesiten controlar de forma remota un entorno EHC.\\
        \hline \textbf{El nombre del producto} & Es un sistema hardware y software de control y gestión de elementos de forma remota. \\
        \hline \textbf{Que} & Proporciona un entorno más agradable al usuario en el hogar, y por lo tanto, afecta a la calidad de vida del usuario.\\
        \hline \textbf{No como} & Las aplicaciones actuales de domotización de hogares donde solo se hacen cargo de realizar peticiones a elementos de casa y con un elevado coste de hardware e instalación. \\
        \hline \textbf{Producto} & Permite gestionar los distintos recursos que ofrezca el lugar donde se encuentre implantado el sistema EHC mediante una interfaz gráfica sencilla y amigable. Además proporciona un acceso rápido y eficiente al sistema desde cualquier punto con acceso a Internet o a través de tareas programadas. \par
        La instalación del sistema en un posible futuro entorno EHC se realiza a un bajo coste y sin necesidad de alterar la estructura del hogar. \\
        \hline
    \end{tabular}

\section{Futuro mercado del sistema}
    La domótica es una herramienta que está creciendo en nuestro país en los últimos años, pero a día de hoy, no está tan extendido como podría estarlo debido a su alto coste tanto económico como de instalación. Los sistemas actuales de domótica necesitan una gran instalación ya que abarcan todos los elementos del hogar y los usuarios se muestran reacios a realizar una obra para domotizar el hogar, por lo que la gran mayoría que dispone de esta herramienta han adquirido el hogar domotizado.
    \par
    El sistema EHC se centra en el usuario que no desee realizar una instalación domótica que implique alterar/modificar la estructura del hogar y a un alto coste económico, por lo que el sistema se mueve en un mercado muy amplio ya que es la manera más fácil de domotizar un hogar.
    \par
    Se ofrece una instalación más cómoda y personalizada con este sistema, ya que   debido a la fácil instalación del sistema los usuarios no se muestran reacios a domotizar el hogar y pueden probar gradualmente a instalar nuestro sistema en el entorno EHC sin necesidad de hacer modificaciones en un lugar concreto.
    \par
    Al contrario que toda empresa de domótica, donde el modelo de negocio de esta empresa se centra en la instalación y mantenimiento de los componentes instalados, el sistema EHC se centra en una instalación a bajo coste y a una suscripción mensual con distintas con distintos perfiles según la suscripción seleccionada por el usuario.
