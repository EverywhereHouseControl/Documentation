\chapter{Introducción}
El plan de desarrollo del proyecto EHC, está caracterizado por su alto contenido hardware, por lo tanto, el modelo de desarrollo, la organización y gestión, y la planificación, tienen distintas peculiaridades, que en ocasiones difieren bastante del modelo de proceso software. Como este proyecto está enmarcado dentro de la asignatura de Ingeniería del Software, se ha buscado siempre, establecer las bases fundamentales lo más semejantes posibles al desarrollo software. Destacar que en el desarrollo de prototipos hardware, muchos de ellos se acaban reutilizando lo que aumenta la productividad por rendimiento asociada.
\section{Propósito}El propósito de este documento es describir el plan de gestión del proyecto, que se pretende seguir durante todo el curso académico de la asignatura. Esta dirigido, tanto a los participantes en el proyecto, como al profesor de la asignatura, que también forma parte importante del proyecto, al ser el que coordinara las entregas actuando como auditor del proyecto. Se presenta a los participantes y se detalla la organización que existe entre ellos.

\section{Ámbito}Este plan de proyecto se enmarca dentro del ámbito universitario, en vista a un año y ligado a la enseñanza del desarrollo software. Formando parte de un proyecto de la facultad de Ingeniería Informática de la Universidad Complutense de Madrid, que se desarrollara durante todo el curso académico 2013/2014.

\section{Referencias}
\begin{itemize}
\item	Documento de Visión. 
\item	Documento de Requisitos. 
\item	Documento de arquitectura del software. 
\item	Documento de plan de pruebas. 
\item	Documento de plan de gestión de riesgos. 

\end{itemize}
\section{Perspectiva general}
El plan de desarrollo de software comprende la siguiente información:
\begin{itemize}
\item	Resumen del proyecto: contiene una descripción del propósito, el alcance y los objetivos del proyecto. También define los elementos que se han de generar durante el desarrollo del proyecto.
\item	Organización del proyecto: describe la estructura organizativa del equipo de desarrollo.
\item	Proceso de gestión: detalla los costes estimados y la planificación, define las fases e hitos del proyecto, y describe el seguimiento del proyecto.
\item	Planes del proceso técnico: proporciona una visión general del proceso de desarrollo de software, incluyendo métodos, herramientas y técnicas a utilizar.
\item	Planes del proceso de soporte: contiene el plan de gestión de la configuración.

\end{itemize}
