\chapter{Visión general}
\section{Propósito, alcance y objetivos del proyecto}
El producto tiene como objetivo, facilitar el uso y la adaptación a los usuarios inexpertos, mediante un acceso rápido y directo, de los dispositivos del hogar. Para una explicación más detallada, revisar el documento de visión.
\section{Suposiciones y restricciones}
Los participantes en el proyecto son en su totalidad estudiantes universitarios. El proyecto carece de presupuesto, más allá de las inversiones a título personal de los integrantes del mismo. El grupo está compuesto por 10 personas, con poca experiencia en trabajo en grupo.
La fecha de finalización del proyecto coincide con la del curso académico 2013/2014.\\

Los conocimientos requeridos son:
\begin{itemize}

\item	Programación dispositivos hardware Arduino y Raspberry PI.
\item	Programación IOS y Android.
\item	Lenguajes de programación de base de datos.
\item	Comunicaciones inalámbricas.
\item	Domótica del hogar, funcionamiento específico.

\end{itemize}
\section{Productos del proyecto}
Se generarán los siguientes productos durante el desarrollo del proyecto:
\begin{itemize}

\item	Plan de desarrollo de software: planificación en la que se pretende apoyar para la elección de las tareas asociadas al sprint.
\item	Plan de gestión de riesgos: documento donde se detallan los planes de prevención, mitigación o contingencia para solventar los posibles riegos durante el proceso de desarrollo del proyecto.
\item	Especificación de Requisitos: formulación de los requisitos, tanto funcionales como no funcionales, que definirá los casos de uso del sistema.
\item	Glosario: definición de términos usados en los documentos del proyecto.
\item	Visión: análisis y definición de las principales características y restricciones del sistema, haciendo hincapié en la funcionalidad requerida por los usuarios finales.
\item	Plan de pruebas: documento que explica el proceso de gestión de calidad y comprobaciones sobre el software producido.
\item	Documento de la Arquitectura: detalla aspectos relativos al diseño y estructura del hardware y del software que intervienen en el proyecto.
\item	Casos de uso: descripción de los distintos usos del sistema. Dominado por la funcionalidad que se ofrece al usuario, definidas éstas en los requisitos.
\item	Planes de iteración: detallan el trabajo y planificación de cada iteración del proyecto. En nuestro caso dirigido a las mejoras incrementales de los prototipos.
\item	Entrega de prototipo ejecutable a finales de enero.
\item	Entrega de la versión final del proyecto en junio.
\end{itemize}