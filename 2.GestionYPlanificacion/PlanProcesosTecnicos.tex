\chapter{Plan de procesos técnicos}
\section{Esquema de desarrollo}
Para el desarrollo del proyecto se ha intentado seguir el modelo SCRUM con prototipos. Este pretende simular la metodología ágil del desarrollo software sobre el hardware. Los prototipos con la piedra angular del proceso y hacia ellos van dirigidos todos los sprint. Así, una vez desarrollados los prototipos básicos, podremos ir reutilizando los y extendiéndoles mejoras sin necesidad de volver a diseñar. Abaratando costes y reduciendo tiempos de producción.\\

Esta sería la idea fundamental, el modelo SCRUM define nuestra forma de trabajar y organizarnos y el desarrollo de prototipos los incrementos sobre el proyecto. En las etapas iniciales se parecerá mucho al SCRUM, y en las finales al desarrollo de prototipos. Esta idea en software podría decirse que se parece a la programación extrema (desarrollo de prototipos).
\section{Métodos, herramientas y técnicas}
La metodología en las funcionalidades hardware, todo lo referente a dispositivos, se basa mucho en la estrategia de prueba y error. Mientras que en la aplicación y el servidor se buscará siempre programación eficiente y clara, conforme a los documentos de estilo que se elaboren.
