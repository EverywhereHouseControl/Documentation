\chapter{Organización}
\section{Estructura organizativa}
La estructura del sistema que se está desarrollando tiene tres partes: parte de Hardware, parte de cliente o Software y parte de Servidor, pero será detallada más adelante.
Debido a esta estructura, se ha escogido una organización de equipo de tipo ``Descentralizado Controlado''.
En esta organización se han definido tres departamentos (Hardware, Software y Servidor), que se encargan de cada uno de los tres pilares del sistema. En cada departamento hay un ``Gestor de Departamento'' que se encarga de que la evolución sea constante.\\

Además se ha definido también el rol de "Documentador", que se encargará de organizar la documentación y de asegurarse de que sea lo más completa y útil posible. Las tareas de este rol se compaginan con las tareas propias de su departamento.
Hay también un ``Gestor Funcional del Sistema'', cuya función es asegurarse del buen funcionamiento de la comunicación entre los departamentos.\\

No obstante tanto el Gestor Funcional del Sistema, como los Gestores de Proyecto como los Documentadores son también desarrolladores, es decir, todo el personal dedicado al proyecto está involucrado activamente en las labores de desarrollo del sistema.\\

Por último, hay que hablar del cliente. Este proyecto no surge porque un tercero haya contactado con una empresa, para contratar un servicio o comprar un producto, sino que tiene un fin didáctico primordialmente. Teniendo esto en cuenta, el cliente será aquella persona/s que acoten, de alguna manera, los requisitos del sistema.
\section{Modelo de proceso}
El modelo de proceso que se sigue en el proyecto es uno basado en prototipos, combinado con una metodología ágil Scrum.
La razón de escoger este modelo es la naturaleza del sistema que se está desarrollando. El núcleo del proyecto no tiene demasiada envergadura,  sino que es en las numerosas funcionalidades donde recae la verdadera complejidad. Como la mayoría de éstas son independientes unas de otras, se pueden desarrollar por separado. Esto permite diseñar ``sprints'' de Scrum, de tal forma que, al acabar cada sprint (o incluso a mitad de él) se tenga un sistema totalmente funcional (reducido respecto al producto final) que es susceptible de ser modificado más adelante, total o parcialmente, mediante alguna otra tarea que mejore o modifique su comportamiento, en otras palabras, se tenga un prototipo.\\

Para organizar el modelo de proceso, se ha hecho coincidir el contenido de los sprints de Scrum con los prototipos que resultan de cada iteración. \\

El número de reuniones semanales se reduce a dos veces por semana por norma general. Esto nos ofrece más agilidad a la hora de desarrollar el sistema ya que se desarrolla en el comienzo en tres módulos bien diferenciados (Software, Hardware y el servidor) y que no necesitan (en ese momento) integrarse en un único sistema. A medida que se va desarrollando cada móduloeste número de reuniones irá aumentando ya que se necesita una mayor comunicación para integrar los distintos módulos en un único sistema.

El Scrum Master del proyecto es Álvaro Morales
\section{Contactos externos}
Stakeholders fuera del personal del proyecto:
\begin{itemize}
\item Federico Peinado (Cliente)
\item Samer Hassan (Cliente)
\end{itemize}

\section{Roles y responsabilidades}
\subsection{Hardware}
\begin{itemize}
\item Héctor Gutiérrez (Gestor de Departamento; Desarrollador)
\item Álvaro Morales (Gestor Funcional del Sistema; Desarrollador)
\item Víctor Ochoa  (Documentador; Desarrollador)
\item Fernando A. Guzmán  (Desarrollador)
\end{itemize}


\subsection{Software}
\begin{itemize}
\item Víctor (Gestor de Departamento; Desarrollador)
\item Miguel A. Maldonado (Desarrollador)
\item Colin Tirado (Documentador; Desarrollador)
\end{itemize}

\subsection{Servidor}
\begin{itemize}
\item Alejandro Ladrón (Gestor de Departamento; Desarrollador)
\item Jose A. Rey (Desarrollador)
\item Luis A. Saavedra (Desarrollador)
\end{itemize}