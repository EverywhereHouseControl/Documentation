\chapter{Requisitos espec\'ificos}
En esta secci\'on se muestra los distintos requisitos que deben de cumplir el sistema. La estructura de esta secci\'on ser\'a la siguiente:
\begin{description}
\item[Interfaces externas:]{Interfaces necesarias para el correcto funcionamiento de los distintos requisitos}
\item[Requisitos funcionales:]{Define una funci\'on del sistema como un conjunto de entradas, acciones y salidas.}
\item[Requisitos no funcionales:]{Requisito que especifica los criterios que pueden usarse para juzgar la operaci\'on del sistema en lugar de sus comportamientos espec\'ificos.}
\end{description}

Las secciones de los requisitos suelen someterse a cambios hasta la entrega del producto, por lo que es conveniente registrar las versiones, autores y fechas del mismo. 

Este documento ser\'a \'util como punto de partida para el equipo de desarrollo con la elaboraci\'on de cada requisito, ya que se define el entorno de cada requisito. 

\section{Interfaces}
Debido a la complejidad del sistema EHC, es necesario definir una serie de interfaces que garantice el correcto funcionamiento entre los tres grandes m\'odulos que conforman el sistema (Software, Hardware y servidor). En el caso particular de las interfaces usadas por parte de la aplicaci\'on web o m\'ovil, es solo necesario definir la interfaz con el usuario y con el servidor que posteriormente controlar\'a los distintos elementos Hardware.

\subsection{Interfaz de usuario}
Los principales objetivos de la interfaz m\'ovil o web con el usuario son los siguientes puntos:
\begin{enumerate}
\item Accesibilidad.
\item Navegaci\'on r\'apida con el m\'inimo n\'umero de pasos para realizar una acci\'on.
\item Entorno agradable evitando sobrecargar la interfaz con muchos elementos.
\end{enumerate}

Respet\'andose esta serie de condiciones, se garantiza una c\'omoda navegaci\'on sobre el sistema para el usuario. Todo esto ser\'a posible gracias al uso de distintos elementos gr\'aficos de los distintos entornos usados (iOS, Android y HTML5). Adem\'as, para que cada usuario tenga una navegaci\'on personalizada dependiendo de como tenga instalado el entorno EHC, se le solicitar\'a un usuario y contrase\~na para que la aplicaci\'on sea cargada de forma \'unica para ese usuario.

\subsection{Interfaz de software y comunicaci\'on con el servidor}
Esta interfaz es cr\'itica en el sistema ya que sin una interfaz definida entre la aplicaci\'on y el servidor, es imposible la comunicaci\'on entre ambas y por lo tanto, el sistema EHC no funcionar\'ia correctamente.
La comunicaci\'on entre la interfaz de software y el servidor debe de estar garantizada por una serie de medidas de seguridad que asegure lo siguiente:

\begin{enumerate}
\item Confidencialidad: impedir la divulgaci\'on de informaci\'on a personas o sistemas no autorizados.
\item Integridad: mantener los datos libres de modificaciones no autorizadas.
\item Disponibilidad: cuando no exista un problema ajeno a la aplicaci\'on relacionado con la conexi\'on a la red, la informaci\'on debe encontrarse a disposici\'on de quienes quieran acceder a ella.
\item Autenticidad: identifica al generador de la informaci\'on. 
\end{enumerate}

\section{Requisitos funcionales}
\begin{tabular}{|p{3cm}|p{4cm}|p{4cm}|p{4cm}|}
\hline \multicolumn{3}{|p{9cm}|}{\textit{Acceso a la aplicaci\'on}} & \textit{Versi\'on 1.0} \\
\hline \textit{Codigo} & \textit{Fecha} & \textit{Autor} & \textit{Grado de necesidad} \\
RF-01 & 09/12/2013 & Tirado, Colin & Alto \\
\hline \textit{Descripci\'on} & \multicolumn{3}{|p{9cm}|}{Identificaci\'on del usuario en el sistema.} \\
\hline \textit{Entrada} & \multicolumn{3}{|p{9cm}|}{Nombre de usuario y contrase\~na del usuario.} \\
\hline \textit{Precondici\'on} & \multicolumn{3}{|p{9cm}|}{Conexi\'on a la red.} \\
\hline \textit{Acciones} & \multicolumn{3}{|p{9cm}|}{
\begin{enumerate}
\item Acceder a la aplicaci\'on.
\item Introducir el nombre de usuario en el campo \textit{Usuario}.
\item Introducir la contrase\~na en el campo \textit{contrase\~na}.
\item Presionar el bot\'on \textit{Acceder}.
\end{enumerate}
} \\
\hline \textit{Postcondici\'on} & \multicolumn{3}{|p{9cm}|}{Se contempla dos situaciones:
\begin{enumerate}
\item Si se identifica con \'exito: se volcar\'a la informaci\'on correspondiente del usuario.
\item En caso contrario: no se aplica.
\end{enumerate}
} \\
\hline \textit{Salida} & \multicolumn{3}{|p{9cm}|}{Se contemplan tres situaciones
\begin{enumerate}
\item Si el usuario no existe: se notificar\'a que el usuario intruducido no est\'a registrado en el sistema.
\item Si se ha introducido de forma err\'onea la contrase\~na: se notificar\'a que se ha introducido mal la contrase\~na.
\item Si se introduce el usuario y contrase\~na correctamente: se mostrar\'a la ventana principal de la aplicaci\'on.
\end{enumerate} } \\ \hline
\end{tabular}
\newline
\begin{tabular}{|p{3cm}|p{4cm}|p{4cm}|p{4cm}|}
\hline \multicolumn{3}{|p{9cm}|}{\textit{Control de luces}} & \textit{Versi\'on 1.0} \\
\hline \textit{Codigo} & \textit{Fecha} & \textit{Autor} & \textit{Grado de necesidad} \\
RF-02 & 12/12/2013 & Tirado, Colin & Alto \\
\hline \textit{Descripci\'on} & \multicolumn{3}{|p{9cm}|}{El usuario podr\'a encender, apagar o regular la intensidad de un aparato de iluminaci\'on} \\
\hline \textit{Entrada} & \multicolumn{3}{|p{9cm}|}{Un valor entre 0 y 100 indicando la intensidad de la luz deseada por el usuario, siendo 0 apagado y 100 encendido con la máxima intensidad } \\
\hline \textit{Precondic\'ion} & \multicolumn{3}{|p{9cm}|}{Conexi\'on establecida en la red.} \\
\hline \textit{Acciones} & \multicolumn{3}{|p{9cm}|}{Insertar acciones de forma enumerada} \\
\hline \textit{Postcondici\'on} & \multicolumn{3}{|p{9cm}|}{No se aplica.} \\
\hline \textit{Salida} & \multicolumn{3}{|p{9cm}|}{Se realizar\'a la acci\'on enviada por el usuario.} \\ \hline
\end{tabular}


%\newline %Si pones este newline se queja
\begin{tabular}{|p{3cm}|p{4cm}|p{4cm}|p{4cm}|}
\hline \multicolumn{3}{|p{9cm}|}{\textit{Consulta del perfil}} & \textit{Versi\'on 1.0} \\
\hline \textit{Codigo} & \textit{Fecha} & \textit{Autor} & \textit{Grado de necesidad} \\
RF-03 & 10/12/2013 & Tirado, Colin & Alto \\
\hline \textit{Descripci\'on} & \multicolumn{3}{|p{9cm}|}{El usuario podr\'a consultar su perfil EHC.} \\
\hline \textit{Entrada} & \multicolumn{3}{|p{9cm}|}{El usuario selecciona el bot\'on de \textit{Perfil} situado en la ventana principal de la aplicaci\'on.} \\
\hline \textit{Precondici\'on} & \multicolumn{3}{|p{9cm}|}{El usuario est\'a identificado en el sistema EHC} \\
\hline \textit{Acciones} & \multicolumn{3}{|p{9cm}|}{
\begin{enumerate}
\item El usuario se identifica en el sistema EHC.
\item Tras identificarse, acceder\'a a la secci\'on \textit{Perfil} de la aplicaci\'on
\end{enumerate}
} \\
\hline \textit{Postcondici\'on} & \multicolumn{3}{|p{9cm}|}{No aplica} \\
\hline \textit{Salida} & \multicolumn{3}{|p{9cm}|}{Se muestra el perfil del usuario.} \\ \hline
\end{tabular}

\newline
\begin{tabular}{|p{3cm}|p{4cm}|p{4cm}|p{4cm}|}
\hline \multicolumn{3}{|p{9cm}|}{\textit{Modificaci\'on del perfil}} & \textit{Versi\'on 1.0} \\
\hline \textit{Codigo} & \textit{Fecha} & \textit{Autor} & \textit{Grado de necesidad} \\
RF-04 & 10/12/2013 & Tirado, Colin & Alto \\
\hline \textit{Descripci\'on} & \multicolumn{3}{|p{9cm}|}{El usuario podr\'a modificar su perfil EHC.} \\
\hline \textit{Entrada} & \multicolumn{3}{|p{9cm}|}{El usuario selecciona el bot\'on correspondiente para modificar su perfil situado en el perfil.} \\
\hline \textit{Precondici\'on} & \multicolumn{3}{|p{9cm}|}{El usuario est\'a identificado en el sistema EHC} \\
\hline \textit{Acciones} & \multicolumn{3}{|p{9cm}|}{
\begin{enumerate}
\item El usuario se identifica en el sistema EHC.
\item Tras identificarse, acceder\'a a la secci\'on \textit{Perfil} de la aplicaci\'on
\item Selecciona el bot\'on de modificar
\end{enumerate}
} \\
\hline \textit{Postcondici\'on} & \multicolumn{3}{|p{9cm}|}{Si se ha realizado alg\'un cambio correctamente, la nueva informaci\'on introducida ser\'a almacenada en nuestra base de datos.} \\
\hline \textit{Salida} & \multicolumn{3}{|p{9cm}|}{Se muestra el perfil del usuario y se notifica si se ha realizado correctamente el cambio.} \\ \hline
\end{tabular}

\newline
\begin{tabular}{|p{3cm}|p{4cm}|p{4cm}|p{4cm}|}
\hline \multicolumn{3}{|p{9cm}|}{\textit{Consulta de eventos programados}} & \textit{Versi\'on 1.0} \\
\hline \textit{Codigo} & \textit{Fecha} & \textit{Autor} & \textit{Grado de necesidad} \\
RF-05 & 10/12/2013 & Tirado, Colin & Alto \\
\hline \textit{Descripci\'on} & \multicolumn{3}{|p{9cm}|}{El usuario podr\'a consultar los eventos programados para el entorno EHC} \\
\hline \textit{Entrada} & \multicolumn{3}{|p{9cm}|}{El usuario selecciona el bot\'on de \textit{Eventos} situado en la ventana principal de la aplicaci\'on.} \\
\hline \textit{Precondici\'on} & \multicolumn{3}{|p{9cm}|}{El usuario est\'a identificado en el sistema EHC} \\
\hline \textit{Acciones} & \multicolumn{3}{|p{9cm}|}{
\begin{enumerate}
\item El usuario se identifica en el sistema EHC.
\item Tras identificarse, acceder\'a a la secci\'on \textit{Eventos} de la aplicaci\'on
\end{enumerate}
} \\
\hline \textit{Postcondici\'on} & \multicolumn{3}{|p{9cm}|}{No aplica.} \\
\hline \textit{Salida} & \multicolumn{3}{|p{9cm}|}{Se mostrará los eventos programados del entorno EHC} \\ \hline
\end{tabular}
\newline
\begin{tabular}{|p{3cm}|p{4cm}|p{4cm}|p{4cm}|}
\hline \multicolumn{3}{|p{9cm}|}{\textit{A\~nadir un evento}} & \textit{Versi\'on 1.0} \\
\hline \textit{Codigo} & \textit{Fecha} & \textit{Autor} & \textit{Grado de necesidad} \\
RF-06 & 10/12/2013 & Tirado, Colin & Alto \\
\hline \textit{Descripci\'on} & \multicolumn{3}{|p{9cm}|}{El usuario podr\'a a\~nadir un evento a su entorno EHC} \\
\hline \textit{Entrada} & \multicolumn{3}{|p{9cm}|}{El usuario selecciona el bot\'on de \textit{Eventos} situado en la ventana principal de la aplicaci\'on.} \\
\hline \textit{Precondici\'on} & \multicolumn{3}{|p{9cm}|}{El usuario est\'a identificado en el sistema EHC.} \\
\hline \textit{Acciones} & \multicolumn{3}{|p{9cm}|}{
\begin{enumerate}
\item El usuario se identifica en el sistema EHC.
\item Tras identificarse, acceder\'a a la secci\'on \textit{Eventos} de la aplicaci\'on
\item Selecciona el bot\'on \textit{A\~nadir evento}.
\end{enumerate}
} \\
\hline \textit{Postcondici\'on} & \multicolumn{3}{|p{9cm}|}{Se a\~nade un nuevo evento asociado al entorno EHC correspondiente.} \\
\hline \textit{Salida} & \multicolumn{3}{|p{9cm}|}{Se notificar\'a al usuario si se ha a\~nadido correctamente el evento.} \\ \hline
\end{tabular}
\newline
\begin{tabular}{|p{3cm}|p{4cm}|p{4cm}|p{4cm}|}
\hline \multicolumn{3}{|p{9cm}|}{\textit{Acceder a la configuraci\'on de la aplicaci\'on EHC}} & \textit{Versi\'on 1.0} \\
\hline \textit{Codigo} & \textit{Fecha} & \textit{Autor} & \textit{Grado de necesidad} \\
RF-03 & 10/12/2013 & Tirado, Colin & Alto \\
\hline \textit{Descripci\'on} & \multicolumn{3}{|p{9cm}|}{El usuario podr\'a acceder a la configuraci\'on de la aplicaci\'on EHC para cambiar el sistema de notificaciones, tama\~no de la fuente...} \\
\hline \textit{Entrada} & \multicolumn{3}{|p{9cm}|}{El usuario selecciona el bot\'on de \textit{Configuraci\'on} situado en la ventana principal de la aplicaci\'on.} \\
\hline \textit{Precondici\'on} & \multicolumn{3}{|p{9cm}|}{El usuario est\'a identificado en el sistema EHC.} \\
\hline \textit{Acciones} & \multicolumn{3}{|p{9cm}|}{
\begin{enumerate}
\item El usuario se identifica en el sistema EHC.
\item Tras identificarse, acceder\'a a la secci\'on \textit{Configuraci\'on} de la aplicaci\'on
\end{enumerate}
} \\
\hline \textit{Postcondici\'on} & \multicolumn{3}{|p{9cm}|}{Se realizar\'a los cambios seleccionados por el usuario.} \\
\hline \textit{Salida} & \multicolumn{3}{|p{9cm}|}{Se notificar\'a al usuario si se ha realizado correctamente los cambios.} \\ \hline
\end{tabular}
\newline
\begin{tabular}{|p{3cm}|p{3cm}|p{3cm}|p{3cm}|}
\hline \multicolumn{3}{|p{9cm}|}{\textit{Consulta metereol\'ogica en el entorno EHC.}} & \textit{Versi\'on 1.0} \\
\hline \textit{Codigo} & \textit{Fecha} & \textit{Autor} & \textit{Grado de necesidad} \\
RF-08 & 10/12/2013 & Tirado, Colin & Medio \\
\hline \textit{Descripci\'on} & \multicolumn{3}{|p{9cm}|}{El usuario podr\'a consultar la metereolog\'ia en el entorno EHC para as\'i, poder realizar alguna acci\'on concreta.} \\
\hline \textit{Entrada} & \multicolumn{3}{|p{9cm}|}{El usuario selecciona el bot\'on de \textit{Consulta metereol\'ogica} situado en la ventana de \textit{Gesti\'on}.} \\
\hline \textit{Precondici\'on} & \multicolumn{3}{|p{9cm}|}{El usuario est\'a identificado en el sistema EHC y tiene configurado un entorno EHC.} \\
\hline \textit{Acciones} & \multicolumn{3}{|p{9cm}|}{
\begin{enumerate}
\item El usuario se identifica en el sistema EHC.
\item Tras identificarse, acceder\'a a la secci\'on \textit{Gesti'on} de la aplicaci\'on.
\item Selecciona el bot\'on de \textit{Consulta metereol\'ogica}.
\end{enumerate}
} \\
\hline \textit{Postcondici\'on} & \multicolumn{3}{|p{9cm}|}{No aplica.} \\
\hline \textit{Salida} & \multicolumn{3}{|p{9cm}|}{Se mostrar\'a la informaci\'on metereol\'ogica del entorno EHC.} \\ \hline
\end{tabular}
\newline
\begin{tabular}{|p{3cm}|p{4cm}|p{4cm}|p{4cm}|}
\hline \multicolumn{3}{|p{9cm}|}{\textit{Controlar una trampilla}} & \textit{Versi\'on 1.0} \\
\hline \textit{Codigo} & \textit{Fecha} & \textit{Autor} & \textit{Grado de necesidad} \\
RF-09 & 10/12/2013 & Tirado, Colin & Medio \\
\hline \textit{Descripci\'on} & \multicolumn{3}{|p{9cm}|}{El usuario podr\'a controlar desde su aplicaci\'on EHC una trampilla instalada en el entorno EHC del usuario.} \\
\hline \textit{Entrada} & \multicolumn{3}{|p{9cm}|}{El usuario selecciona el bot\'on correspondiente situado en la zona donde se encuentra dicho elemento en el entorno EHC.} \\
\hline \textit{Precondici\'on} & \multicolumn{3}{|p{9cm}|}{
\begin{itemize}
\item El usuario est\'a identificado en el sistema EHC
\item Tiene configurado un entorno EHC.
\item Tiene instalado una trampilla en el entorno EHC
\end{itemize}
} \\
\hline \textit{Acciones} & \multicolumn{3}{|p{9cm}|}{
\begin{enumerate}
\item El usuario se identifica en el sistema EHC.
\item Tras identificarse, acceder\'a a la secci\'on \textit{Gesti\'on} de la aplicaci\'on.
\item Selecciona la zona del entorno EHC donde desea controlar el elemento.
\item Realiza la acci\'on sobre el elemento a controlar.
\end{enumerate}
} \\
\hline \textit{Postcondici\'on} & \multicolumn{3}{|p{9cm}|}{Cambia el estado de la trampilla en el sistema EHC.} \\
\hline \textit{Salida} & \multicolumn{3}{|p{9cm}|}{Se realizar\'a la acci\'on enviada por el usuario.} \\ \hline
\end{tabular}
\newline
\begin{tabular}{|p{3cm}|p{4cm}|p{4cm}|p{4cm}|}
\hline \multicolumn{3}{|p{9cm}|}{\textit{Control remoto de la climatizaci\'on}} & \textit{Versi\'on 1.0} \\
\hline \textit{Codigo} & \textit{Fecha} & \textit{Autor} & \textit{Grado de necesidad} \\
RF-10 & 10/12/2013 & Tirado, Colin & Alto \\
\hline \textit{Descripci\'on} & \multicolumn{3}{|p{9cm}|}{El usuario podr\'a controlar la climatizaci\'on del hogar.} \\
\hline \textit{Entrada} & \multicolumn{3}{|p{9cm}|}{El usuario selecciona el bot\'on correspondiente situado en la zona donde se encuentra dicho elemento en el entorno EHC.} \\
\hline \textit{Precondici\'on} & \multicolumn{3}{|p{9cm}|}{
\begin{itemize}
\item El usuario est\'a identificado en el sistema EHC
\item Tiene configurado un entorno EHC.
\item Tiene instalado un sistema de climatizaci\'on en el entorno EHC.
\end{itemize}
} \\
\hline \textit{Acciones} & \multicolumn{3}{|p{9cm}|}{
\begin{enumerate}
\item El usuario se identifica en el sistema EHC.
\item Tras identificarse, acceder\'a a la secci\'on \textit{Gesti\'on} de la aplicaci\'on.
\item Selecciona la zona del entorno EHC donde desea controlar el elemento.
\item Realiza la acci\'on sobre el elemento a controlar.
\end{enumerate}
} \\
\hline \textit{Postcondici\'on} & \multicolumn{3}{|p{9cm}|}{Cambia el estado de la climatizaci\'on en el sistema EHC.} \\
\hline \textit{Salida} & \multicolumn{3}{|p{9cm}|}{Se realizar\'a la acci\'on enviada por el usuario.} \\ \hline
\end{tabular}
\newline
\begin{tabular}{|p{3cm}|p{4cm}|p{4cm}|p{4cm}|}
\hline \multicolumn{3}{|p{9cm}|}{\textit{Control del tel\'efono fijo}} & \textit{Versi\'on 1.0} \\
\hline \textit{Codigo} & \textit{Fecha} & \textit{Autor} & \textit{Grado de necesidad} \\
RF-11 & 10/12/2013 & Tirado, Colin & Medio \\
\hline \textit{Descripci\'on} & \multicolumn{3}{|p{9cm}|}{El usuario podr\'a controlar desde la aplicaci\'on EHC el tel\'efono fijo, pudiendo recibir o realizar llamadas.} \\
\hline \textit{Entrada} & \multicolumn{3}{|p{9cm}|}{Se contempla dos situaciones
\begin{enumerate}
\item Si desea realizar una llamada: el usuario selecciona el bot\'on correspondiente situado en la zona donde se encuentra dicho elemento en el entorno EHC.
\item Si recibe una llamada: el usuario selecciona el aviso de la aplicaci\'on EHC y selecciona si desea aceptar o rechazar la llamada.
\end{enumerate}} \\
\hline \textit{Precondici\'on} & \multicolumn{3}{|p{9cm}|}{
\begin{itemize}
\item El usuario est\'a identificado en el sistema EHC
\item Tiene configurado un entorno EHC.
\item Tiene instalado un tel\'efono fijo en el entorno EHC
\end{itemize}
} \\
\hline \textit{Acciones} & \multicolumn{3}{|p{9cm}|}{
Para el caso querer realizar una llamada:
\begin{enumerate}
\item El usuario se identifica en el sistema EHC.
\item Tras identificarse, acceder\'a a la secci\'on \textit{Gesti\'on} de la aplicaci\'on.
\item Selecciona la zona del entorno EHC donde desea controlar el elemento.
\item Realiza la acci\'on sobre el elemento a controlar.
\end{enumerate}

Si desea aceptar/rechazar una llamada entrante
\begin{enumerate}
\item El usuario recibe un aviso de una llamada entrante a su entorno EHC.
\item El usuario selecciona si desea rechazar o aceptar una llamada entrante.
\end{enumerate}
} \\
\hline \textit{Postcondici\'on} & \multicolumn{3}{|p{9cm}|}{No aplica.} \\
\hline \textit{Salida} & \multicolumn{3}{|p{9cm}|}{Se mostrar\'a la duracci\'on de la llamada realizada en el caso de que se realize una llamda. En otro caso, no se muestra nada m\'as.} \\ \hline
\end{tabular}
\newline
\begin{tabular}{|p{3cm}|p{4cm}|p{4cm}|p{4cm}|}
\hline \multicolumn{3}{|p{9cm}|}{\textit{Sistema de alarmas}} & \textit{Versi\'on 1.0} \\
\hline \textit{Codigo} & \textit{Fecha} & \textit{Autor} & \textit{Grado de necesidad} \\
RF-12 & 10/12/2013 & Tirado, Colin & Alto \\
\hline \textit{Descripci\'on} & \multicolumn{3}{|p{9cm}|}{El usuario recibir\'a un aviso de su entorno EHC cuando el sistema de alarmas ha recibido una situaci\'on extra\~na.} \\
\hline \textit{Entrada} & \multicolumn{3}{|p{9cm}|}{No aplica.} \\
\hline \textit{Precondici\'on} & \multicolumn{3}{|p{9cm}|}{
\begin{itemize}
\item El usuario est\'a identificado en el sistema EHC.
\item Tiene configurado un entorno EHC.
\item Tiene instalado un sistema de alarmas en el entorno EHC.
\end{itemize}
} \\
\hline \textit{Acciones} & \multicolumn{3}{|p{9cm}|}{No aplica.} \\
\hline \textit{Postcondici\'on} & \multicolumn{3}{|p{9cm}|}{Se almacena la informaci\'on relativa al aviso enviado.} \\
\hline \textit{Salida} & \multicolumn{3}{|p{9cm}|}{No aplica.} \\ \hline
\end{tabular}
\newline
\begin{tabular}{|p{3cm}|p{4cm}|p{4cm}|p{4cm}|}
\hline \multicolumn{3}{|p{9cm}|}{\textit{Control remoto de la puerta del garaje.}} & \textit{Versi\'on 1.0} \\
\hline \textit{Codigo} & \textit{Fecha} & \textit{Autor} & \textit{Grado de necesidad} \\
RF-13 & 10/12/2013 & Tirado, Colin & Media \\
\hline \textit{Descripci\'on} & \multicolumn{3}{|p{9cm}|}{El usuario podr\'a controlar la puerta del garaje.} \\
\hline \textit{Entrada} & \multicolumn{3}{|p{9cm}|}{El usuario selecciona el bot\'on correspondiente situado en la zona donde se encuentra dicho elemento en el entorno EHC.} \\
\hline \textit{Precondici\'on} & \multicolumn{3}{|p{9cm}|}{
\begin{itemize}
\item El usuario est\'a identificado en el sistema EHC.
\item Tiene configurado un entorno EHC.
\item Tiene instalado un garaje autom\'atico en el entorno EHC.
\end{itemize}
} \\
\hline \textit{Acciones} & \multicolumn{3}{|p{9cm}|}{
\begin{enumerate}
\item El usuario se identifica en el sistema EHC.
\item Tras identificarse, acceder\'a a la secci\'on \textit{Gesti\'on} de la aplicaci\'on.
\item Selecciona la zona del entorno EHC donde desea controlar el elemento.
\item Realiza la acci\'on sobre el elemento a controlar.
\end{enumerate}
} \\
\hline \textit{Postcondici\'on} & \multicolumn{3}{|p{9cm}|}{Cambia el estado de  en el sistema EHC.} \\
\hline \textit{Salida} & \multicolumn{3}{|p{9cm}|}{Se realizar\'a la acci\'on enviada por el usuario.} \\ \hline
\end{tabular}
\newline
\begin{tabular}{|p{3cm}|p{4cm}|p{4cm}|p{4cm}|}
\hline \multicolumn{3}{|p{9cm}|}{\textit{Control de dispositivos por inflarojos}} & \textit{Versi\'on 1.0} \\
\hline \textit{Codigo} & \textit{Fecha} & \textit{Autor} & \textit{Grado de necesidad} \\
RF-14 & 10/12/2013 & Tirado, Colin & Alto \\
\hline \textit{Descripci\'on} & \multicolumn{3}{|p{9cm}|}{El usuario podr\'a controlar dispositivos que sean controlados a trav\'es de inflarojos. Estos dispositivos pueden ser televisiones, equipos de m\'usica, reproductores de DVD...} \\
\hline \textit{Entrada} & \multicolumn{3}{|p{9cm}|}{El usuario selecciona el bot\'on correspondiente situado en la zona donde se encuentra dicho elemento en el entorno EHC.} \\
\hline \textit{Precondici\'on} & \multicolumn{3}{|p{9cm}|}{
\begin{itemize}
\item El usuario est\'a identificado en el sistema EHC.
\item Tiene configurado un entorno EHC.
\item Tiene instalado y configurado alg\'un elemento en el entorno EHC.
\end{itemize}
} \\
\hline \textit{Acciones} & \multicolumn{3}{|p{9cm}|}{
\begin{enumerate}
\item El usuario se identifica en el sistema EHC.
\item Tras identificarse, acceder\'a a la secci\'on \textit{Gesti\'on} de la aplicaci\'on.
\item Selecciona la zona del entorno EHC donde desea controlar el elemento.
\item Realiza la acci\'on sobre el elemento a controlar.
\end{enumerate}
} \\
\hline \textit{Postcondici\'on} & \multicolumn{3}{|p{9cm}|}{Cambia el estado de  en el sistema EHC.} \\
\hline \textit{Salida} & \multicolumn{3}{|p{9cm}|}{Se realizar\'a la acci\'on enviada por el usuario.} \\ \hline
\end{tabular}
\newline
\begin{tabular}{|p{3cm}|p{4cm}|p{4cm}|p{4cm}|}
\hline \multicolumn{3}{|p{9cm}|}{\textit{Control de persianas}} & \textit{Versi\'on 1.0} \\
\hline \textit{Codigo} & \textit{Fecha} & \textit{Autor} & \textit{Grado de necesidad} \\
RF-15 & 10/12/2013 & Tirado, Colin & Alto \\
\hline \textit{Descripci\'on} & \multicolumn{3}{|p{9cm}|}{El usuario podr\'a controlar las persianas del entorno EHC.} \\
\hline \textit{Entrada} & \multicolumn{3}{|p{9cm}|}{El usuario selecciona el bot\'on correspondiente situado en la zona donde se encuentra dicho elemento en el entorno EHC.} \\
\hline \textit{Precondici\'on} & \multicolumn{3}{|p{9cm}|}{
\begin{itemize}
\item El usuario est\'a identificado en el sistema EHC.
\item Tiene configurado un entorno EHC.
\item Tiene instalado y configurado las persianas para el entorno EHC.
\end{itemize}
} \\
\hline \textit{Acciones} & \multicolumn{3}{|p{9cm}|}{
\begin{enumerate}
\item El usuario se identifica en el sistema EHC.
\item Tras identificarse, acceder\'a a la secci\'on \textit{Gesti\'on} de la aplicaci\'on.
\item Selecciona la zona del entorno EHC donde desea controlar el elemento.
\item Realiza la acci\'on sobre el elemento a controlar.
\end{enumerate}
} \\
\hline \textit{Postcondici\'on} & \multicolumn{3}{|p{9cm}|}{Cambia el estado de  en el sistema EHC.} \\
\hline \textit{Salida} & \multicolumn{3}{|p{9cm}|}{Se realizar\'a la acci\'on enviada por el usuario.} \\ \hline
\end{tabular}
\newline
\begin{tabular}{|p{3cm}|p{4cm}|p{4cm}|p{4cm}|}
\hline \multicolumn{3}{|p{9cm}|}{\textit{Control de una regleta.}} & \textit{Versi\'on 1.0} \\
\hline \textit{Codigo} & \textit{Fecha} & \textit{Autor} & \textit{Grado de necesidad} \\
RF-16 & 10/12/2013 & Tirado, Colin & Alto \\
\hline \textit{Descripci\'on} & \multicolumn{3}{|p{9cm}|}{El usuario podr\'a apagar o encender dispositivos conectados a una regleta.} \\
\hline \textit{Entrada} & \multicolumn{3}{|p{9cm}|}{El usuario selecciona el bot\'on correspondiente situado en la zona donde se encuentra dicho elemento en el entorno EHC.} \\
\hline \textit{Precondici\'on} & \multicolumn{3}{|p{9cm}|}{
\begin{itemize}
\item El usuario est\'a identificado en el sistema EHC.
\item Tiene configurado un entorno EHC.
\item Tiene instalado y configurado una regleta para el entorno EHC.
\end{itemize}
} \\
\hline \textit{Acciones} & \multicolumn{3}{|p{9cm}|}{
\begin{enumerate}
\item El usuario se identifica en el sistema EHC.
\item Tras identificarse, acceder\'a a la secci\'on \textit{Gesti\'on} de la aplicaci\'on.
\item Selecciona la zona del entorno EHC donde desea controlar el elemento.
\item Realiza la acci\'on sobre el elemento a controlar.
\end{enumerate}
} \\
\hline \textit{Postcondici\'on} & \multicolumn{3}{|p{9cm}|}{Cambia el estado de  en el sistema EHC.} \\
\hline \textit{Salida} & \multicolumn{3}{|p{9cm}|}{Se realizar\'a la acci\'on enviada por el usuario.} \\ \hline
\end{tabular}
\newline
\begin{tabular}{|p{3cm}|p{4cm}|p{4cm}|p{4cm}|}
\hline \multicolumn{3}{|p{9cm}|}{\textit{Mostrar texto en un display}} & \textit{Versi\'on 1.0} \\
\hline \textit{Codigo} & \textit{Fecha} & \textit{Autor} & \textit{Grado de necesidad} \\
RF-17 & 10/12/2013 & Tirado, Colin & Alto \\
\hline \textit{Descripci\'on} & \multicolumn{3}{|p{9cm}|}{El usuario podr\'a dejar un mensaje en un display en el entorno EHC.} \\
\hline \textit{Entrada} & \multicolumn{3}{|p{9cm}|}{El usuario selecciona el bot\'on correspondiente situado en la zona donde se encuentra dicho elemento en el entorno EHC.} \\
\hline \textit{Precondici\'on} & \multicolumn{3}{|p{9cm}|}{
\begin{itemize}
\item El usuario est\'a identificado en el sistema EHC.
\item Tiene configurado un entorno EHC.
\item Tiene instalado y configurado un display en el entorno EHC.
\end{itemize}
} \\
\hline \textit{Acciones} & \multicolumn{3}{|p{9cm}|}{
\begin{enumerate}
\item El usuario se identifica en el sistema EHC.
\item Tras identificarse, acceder\'a a la secci\'on \textit{Gesti\'on} de la aplicaci\'on.
\item Selecciona la zona del entorno EHC donde desea controlar el elemento.
\item Selecciona el display.
\item Introduce el texto y presiona el bot\'on \textit{Enviar}.
\end{enumerate}
} \\
\hline \textit{Postcondici\'on} & \multicolumn{3}{|p{9cm}|}{Env\'ia el mensaje introducido al usuario al servidor.} \\
\hline \textit{Salida} & \multicolumn{3}{|p{9cm}|}{Se realizar\'a la acci\'on enviada por el usuario.} \\ \hline
\end{tabular}
\newline
\begin{tabular}{|p{3cm}|p{4cm}|p{4cm}|p{4cm}|}
\hline \multicolumn{3}{|p{9cm}|}{\textit{Control del riego}} & \textit{Versi\'on 1.0} \\
\hline \textit{Codigo} & \textit{Fecha} & \textit{Autor} & \textit{Grado de necesidad} \\
RF-18 & 10/12/2013 & Tirado, Colin & Alto \\
\hline \textit{Descripci\'on} & \multicolumn{3}{|p{9cm}|}{El usuario podr\'a controlar el riego instalado en el entorno EHC .} \\
\hline \textit{Entrada} & \multicolumn{3}{|p{9cm}|}{El usuario selecciona el bot\'on correspondiente situado en la zona donde se encuentra dicho elemento en el entorno EHC.} \\
\hline \textit{Precondici\'on} & \multicolumn{3}{|p{9cm}|}{
\begin{itemize}
\item El usuario est\'a identificado en el sistema EHC.
\item Tiene configurado un entorno EHC.
\item Tiene instalado y configurado un sistema de riego autom\'atico en el entorno EHC.
\end{itemize}
} \\
\hline \textit{Acciones} & \multicolumn{3}{|p{9cm}|}{
\begin{enumerate}
\item El usuario se identifica en el sistema EHC.
\item Tras identificarse, acceder\'a a la secci\'on \textit{Gesti\'on} de la aplicaci\'on.
\item Selecciona la zona del entorno EHC donde desea controlar el elemento.
\item Realiza la acci\'on sobre el elemento a controlar.
\end{enumerate}
} \\
\hline \textit{Postcondici\'on} & \multicolumn{3}{|p{9cm}|}{Env\'ia la petici\'on solicitada por el usuario al servidor del sistema} \\
\hline \textit{Salida} & \multicolumn{3}{|p{9cm}|}{Se realizar\'a la acci\'on enviada por el usuario.} \\ \hline
\end{tabular}
\newline

\section{Requisitos no funcionales}