\begin{tabular}{|p{3cm}|p{4cm}|p{4cm}|p{4cm}|}
\hline \multicolumn{3}{|p{9cm}|}{\textit{A\~nadir un evento}} & \textit{Versi\'on 1.0} \\
\hline \textit{Codigo} & \textit{Fecha} & \textit{Autor} & \textit{Grado de necesidad} \\
RF-06 & 10/12/2013 & Tirado, Colin & Alto \\
\hline \textit{Descripci\'on} & \multicolumn{3}{|p{9cm}|}{El usuario podr\'a a\~nadir un evento a su entorno EHC} \\
\hline \textit{Entrada} & \multicolumn{3}{|p{9cm}|}{El usuario selecciona el bot\'on de \textit{Eventos} situado en la ventana principal de la aplicaci\'on.} \\
\hline \textit{Precondici\'on} & \multicolumn{3}{|p{9cm}|}{El usuario est\'a identificado en el sistema EHC.} \\
\hline \textit{Acciones} & \multicolumn{3}{|p{9cm}|}{
\begin{enumerate}
\item El usuario se identifica en el sistema EHC.
\item Tras identificarse, acceder\'a a la secci\'on \textit{Eventos} de la aplicaci\'on
\item Selecciona el bot\'on \textit{A\~nadir evento}.
\end{enumerate}
} \\
\hline \textit{Postcondici\'on} & \multicolumn{3}{|p{9cm}|}{Se a\~nade un nuevo evento asociado al entorno EHC correspondiente.} \\
\hline \textit{Salida} & \multicolumn{3}{|p{9cm}|}{Se notificar\'a al usuario si se ha a\~nadido correctamente el evento.} \\ \hline
\end{tabular}